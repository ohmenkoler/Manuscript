\chapter{Circuit d'eau fermé pour l'échangeur ambiant}

Pour éviter la consommation excessive d'eau lors des campagnes de mesure, il est choisi d'assurer le maintien de la température de l'échangeur ambiant au moyen d'un circuit d'eau fermé. Ce circuit doit répondre à plusieurs critères :

\begin{enumerate}
    \item L'eau doit être stockée dans un reservoir dont le volume assure une augmentation de moins d'\qty{1}{\degreeCelsius} sur \qty{4}{\hour} de la température de l'eau ;\label{list:CircFerme_Volume}%
    \item le débit d'eau doit à la fois permettre d'extraire toute la chaleur à l'échangeur ambiant tout en réduisant les incertitudes de mesures liées aux capteurs de température de l'eau ;\label{list:CircFerme_Debit}%
    \item la pompe qui alimente le circuit doit \textcolor{red}{disposer/offrir} une charge suffisante pour compenser les pertes de charges régulière et singulières ainsi que la différence d'altitude entre l'échangeur et le reservoir d'eau.\label{list:CircFerme_Charge}
\end{enumerate}

\section{Débit}
La question du débit d'eau a été traitée dans l'annexe~\ref{chap:AHX}-\nameref{chap:AHX}, il est donc retenu un débit de $\qty{7}{\litre\per\minute}$

\section{Charges à fournir}
Pour choisir la pompe adaptée au circuit d'eau, 

\subsection{Différence de hauteur}
Le réservoir doit être disposé dans une pièce située sous le prototype du réfrigérateur. La hauteur à monter est de \qty{3}{\meter}.

\subsection{Régulières}
La connexion de l'échangeur au réservoir et à la pompe se fait par une trappe existante située à une dizaine de mètres du réfrigérateur. Pour évaluer les pertes de charge régulières causées par les tubes de raccordement, \qty{15}{\metre} de tube de diamètre 

\subsection{Singulières}


\section{Volume d'eau}
