\chapter{Protocole expérimental général}
\mylocaltoc


\section{Introduction du chapitre}\label{chap:IntroProtocExp}
Après avoir brièvement rappelé les bases de fonctionnement des machines thermoacoustiques, il est temps de décrire le réfrigérateur support de cette thèse. 

Le choix de la géométrie, les paramètres hydrauliques du régénérateur utilisé, et enfin la chaîne d'excitation et d'acquisition sont présentés dans la section~\ref{chap:PresentationTacot} (\nameref{chap:PresentationTacot}). La deuxième partie présente les conditions expérimentales choisies ainsi que le protocole suivi pour chaque mesure dans la section~\ref{chap:ProtocolExpe} (\nameref{chap:ProtocolExpe}). Les simulations théoriques réalisées sont ensuite expliquées en troisième partie, dans la section~\ref{chap:SimusRealisees} (\nameref{chap:SimusRealisees}).

Ces parties visent à créer une vue d'ensemble de ce qui est fait durant cette thèse et le présenter de manière globale pour pouvoir s'y référer dans les différents chapitres suivants.

\section{Présentation du dispositif expérimental actuel}\label{chap:PresentationTacot}
\subsection{Géométrie du réfrigérateur \textsc{Tacot}}
\subsubsection{Cavité thermoacoustique}
La pompe à chaleur a été dimensionnée et fabriquée dans le cadre du projet ANR \textsc{Tacot} (ThermoAcoustic Cooler for Onroad Transportation), qui porte sur l'application d'une pompe à chaleur thermoacoustique pour la climatisation automobile \cite{ANR_thermo-acoustic_2019}. Ce projet apporte beaucoup de contraintes, dont l'une des principales est la compacité. Contrairement aux autres systèmes thermoacoustiques existant et bien plus volumineux (tels que le liquéfacteur de gaz naturel développé par Swift et Wollan au Los Alamos National Laboratory \cite{swift_thermoacoustics_2002, wollan_development_2002}, ou le réfrigérateur cryogénique thermoacoustique spatial (STAR) \cite{adeff_measurement_1991, garrett_thermoacoustic_1993}), les dimensions doivent être réduites tout en conservant un pompage de chaleur efficace. Pour cela, une géométrie coaxiale pour la cavité thermoacoustique est préférée à celle toroïdale usuellement utilisée en suivant les travaux de Poignand \textit{et al.} \cite{poignand_thermoacoustic_2011, poignand_analysis_2013}. L'ajout d'une source acoustique secondaire dans la cavité thermoacoustique permet également de gagner en compacité, en remplaçant un résonateur plus long par la masse de son équipage mobile et la souplesse de sa suspension, tel que réalisé dans les travaux de Poese \textit{et al.} \cite{poese_thermoacoustic_2004}. De plus, la présence de cette source assure le déphasage optimal entre pression et vitesse acoustiques au sein du noyau thermoacoustique. Un schéma général présente la géométrie de la pompe à chaleur sur la figure~\ref{fig:SchemaGeneralTACOT}, adapté de Ramadan \textit{et al.} \cite{ramadan_design_2021}.


\begin{figure}[!ht]
    \centering
    \external{fig_SchemaGeneralTACOT}
%    \externalremake
    \begin{tikzpicture}[scale=.5]
	\node[anchor=south west, inner sep=0] (image) at (0,0) {\includegraphics[angle=0,origin=c,width=.6\textwidth]{../fig/fig_TACOTSchematics/TACOT.png}};
	
	\begin{scope}[x={(image.south east)},y={(image.north west)}]
	
%		\filldraw (0,0) circle (2pt); 
%		\filldraw[green] (1,0) circle (1pt);
%		\filldraw[red] (1,1) circle (1pt);		
%		\filldraw[blue] (0,1) circle (1pt);
%		\draw[help lines,xstep=.1,ystep=.1] (0,0) grid (1,1);
%		\fill[orange, rounded corners, opacity=1,draw=orange] (.46,.65) -- ++(132:.09) -- ++(0,-.44) -- ++(48:.09) -- cycle;
		\draw[MatlabYellow,rounded corners,very thick,preaction={fill=MatlabYellow!20,opacity=.5}] (.46,.65) -- ++(132:.09) -- ++(-.03,0) -- ++(0,-.44) --++(.03,0) -- ++(48:.09) -- cycle; %node[left,pos=.5,label={[rotate=90]center:Cavité}]{};
		\draw[MatlabYellow] (.415,0.5) node [label={[rotate=90]center:\textbf{Cône}}]{};
		
%		\draw[blue] (.5,.5) node [anchor=center, preaction={fill=black!20,opacity=.7}] {RIX};
%		\draw[red] (.33,.5) node [anchor=center, preaction={fill=black!20,opacity=.7}] {TA core};	
	
		\draw[MatlabPurple,rounded corners,very thick,preaction={fill=MatlabPurple!20,opacity=.5}] (.365,.7) rectangle (.245,.3) node[pos=.5,label={[rotate=90]center:\textbf{Noyau}}]{};
		
		\node (AHX) at (.27,.65) {};
		\node (Reg) at (.32,.65) {};
		\node (CHX) at (.36,.65) {};
		\node (RIX) at (.77,.4) {};
		\node (HP) at (.19,.4) {};
		
		\draw[<-,very thick,MatlabOrange] (AHX.center) to[out=90,in=0] ($(AHX)+(-.15,.4)$) node[left]{\'Echangeur de chaleur ambiant};		
		\draw[<-,very thick] (Reg.center) to ($(Reg)+(0,.4)$) node[above]{Régénérateur};
		\draw[<-,very thick,MatlabBlue] (CHX.center) to[out=90,in=180] ($(CHX)+(.15,.4)$) node[right]{\'Echangeur de chaleur froid};
		
		\draw[->,very thick,green!50!black] ($(RIX)+(0,-.4)$) -- (RIX.center) node[pos=0,anchor=north]{Source acoustique principale};
		\draw[->,very thick,green!50!black] ($(HP)+(0,-.4)$) -- (HP.center) node[pos=0,anchor=north]{Source acoustique secondaire};
		
%		\draw [white] (.455,.5) node{+};
%		\draw [white] (.41,.65) node{+};
%		\draw [white] (.41,.5) node{+};
%		\draw [white] (.41,.35) node{+};

	\end{scope}

	
\end{tikzpicture}
    \caption{Schéma général du réfrigérateur \textsc{Tacot}.}
    \label{fig:SchemaGeneralTACOT}
\end{figure}



\subsubsection{Noyau thermoacoustique}
Tout comme la machine qui le contient, le noyau adopte une géométrie cylindrique et est composé d'un régénérateur encadré par deux échangeurs de chaleur. Les axes $\mathbf e_x$ et $\mathbf e_r$ sont alors respectivement définis dans les direction axiale et radiale du noyau, avec pour direction positive choisie dans le sens de l'échangeur froid vers l'échangeur ambiant pour le premier, et du centre du noyau vers l'extérieur pour le second.\medskip

Le régénérateur est composé de tissus métalliques (Gantois, modèle : 102045) empilés dans une enceinte cylindrique de diamètre intérieur $D_{\sf reg}=\qty{148}{\mm}$ et de longueur $L_{\sf reg}=\qty{39}{\mm}$ pour atteindre une porosité $\Phi$ souhaitée et définie par 

\begin{align}
	\Phi &= \frac{V_{\sf gaz}}{V_{\sf tot}}, \label{eq:Porosite_Volume}%\\
%	&= \frac{V_{\sf tot}-V_{\sf metal}}{V_{\sf tot}} \nonumber\\
%	&= 1 - \frac{m_{\sf metal}}{m_{\sf tot}} \label{eq:Porosite_Masse}
\end{align}
où $V_{\sf gaz}$ représente le volume occupé par le gaz dans le régénérateur, et $V_{\sf tot}$ le volume total du régénérateur. Le matériau est poreux et tortueux, et le rayon hydraulique est alors défini par \cite{swift_thermoacoustics_2017} %\echaf{ajouter source pour justifier que les matériaux type "mousse" se comporte comme du cylindrique : UPDATE dans Swift TA unifying... chapitre 7 (tortuous porous media)}

\begin{equation}
	r_h = d_w\frac{\Phi}{4(1-\Phi)},
	\label{eq:DefRayonHydrauGantois}
\end{equation}
avec $d_w$ le diamètre du fil. Ce matériau poreux dispose d'une certaine capacité à laisser passer un écoulement. Cette perméabilité est notée $K_p$ et est définie par 

\begin{equation}
	K_p = \echaf{\frac{4 r_h^2 \Phi}{8}},
	\label{eq:DefPermeabilite_LISN}
\end{equation}
\echaf{d'après le excel du LISN}, ou

\begin{equation}
	K_p = \echaf{v_{\sf ref}\frac{\nu \Delta x}{\Delta P}},
	\label{eq:DefPermeabilite_Wikipedia}
\end{equation}
\echaf{d'après la formulation générale.}

Les dimensions et paramètres du régénérateur sont résumés dans le tableau \ref{tab:ParamHydrauTAC}.

\begin{table}[!ht]
    \caption{Paramètres hydrauliques du régénérateur à la fréquence de fonctionnement, \linebreak $f=\qty{47}{\Hz}$}
    \label{tab:ParamHydrauTAC}
    \centering
    \begin{tabular}{l@{\hspace{1cm}}l}
    	\hline
    	\textbf{Paramètre [unité]} & \textbf{Valeur} \\\hline\hline
    	Diamètre du noyau $D_{\sf reg}$ [\unit{\meter}] & \num{148e-3} \\
    	Longueur du régénérateur $L_{\sf reg}$ [\unit{\meter}] & \num{39e-3} \\
    	Diamètre du fil $d_w$ [\unit{\meter}] & \num{53e-6} \\
        Rayon hydraulique $r_h$ [\unit{\meter}] & \num{2.81e-5} \\
        Porosité du noyau $\Phi$ [\unit{\percent}] & \num{68}\\
        Couche limite thermique $\delta_\kappa$ [\unit{\meter}] & \num{1.4452e-4} \\
        Couche limite visqueuse $\delta_\nu$ [\unit{\meter}] & \num{9.1120e-5} \\
        \echaf{Perméabilité} [\unit{\meter\squared}] & \num{2.68e-10} \\
        \hline
    \end{tabular}
\end{table}

Cette machine nécessite entre autres choses le respect de la condition $\delta_{\kappa,\nu} \gg r_h$, de sorte à avoir un excellent contact thermique entre le fluide et le solide poreux. Pour le fluide considéré, les épaisseurs de couches limites sont tracées en fonction de la fréquence et comparé au rayon hydraulique sur la figure~\ref{fig:dKdV}.

\begin{figure}[!ht]
    \centering
    \external{fig_dKdV}
%    \externalremake
    \begin{tikzpicture}
    \def\width{.9*\textwidth};
    \def\height{.45*\width};
    \def\spx{.25cm};
    \def\spy{1.25cm};
    \def\legx{.5cm};
    \def\legy{\legx};
    \def\prop{.45};
    \def\xcursor{47};
    \def\ycursorV{9.112e-5};
    \def\ycursorK{1.4452e-4};
    \def\rh{2.81e-5};
    
    \begin{axis}[name=dKdV,width={\width},height={\height},
    grid=both, minor tick num=10, 
    grid style={line width=.1pt, draw=gray!10},
    major grid style={line width=.2pt,draw=gray!50},
    xlabel={Fréquence $f$ (\unit{\Hz})},
    ylabel={\'Epaisseurs de couches limites $\delta_{\kappa,\nu}$ (\unit{\m})},
    xmin=0,xmax=75,ymin=0,ymax=.5/1000,
    xtick={0,25,50,75,100},
    extra x ticks={47},
    extra x tick style={
        grid=major,
        xticklabel={\num{47}},
        xticklabel style={yshift=0, anchor=north}
        },
    extra y ticks={\rh},
    extra y tick style={
        grid=major,
        yticklabel={$r_h$},
        yticklabel style={yshift=1mm, anchor=east}
        },
    ytick={0,1e-4,...,10e-4},
%    ytick={0,2.81/100000,9.1120/100000,1.4452/10000,
%    	2.5/10000,5/10000,1/1000},
    scaled y ticks = false,
    domain=0:100,
    legend cell align={left},
    legend style = {at={($(1,1)+(-2mm,-2mm)$)},anchor = north east,rounded corners}
    ]
        \addplot[solid,ultra thick,draw=Plasma1] file {../fig/fig_dKdV/data/data_dK.txt};
        \addplot[solid,ultra thick,draw=Plasma64] file {../fig/fig_dKdV/data/data_dV.txt};
        \filldraw[Plasma1] (\xcursor,\ycursorK) circle (2pt) node[above right]{$\delta_\kappa=\qty{\ycursorK}{\meter}$};
        \filldraw[Plasma64] (\xcursor,\ycursorV) circle (2pt) node[below left]{$\delta_\nu=\qty{\ycursorV}{\meter}$};
%        \draw[dashed,black!50] (\xcursor,0) -- (\xcursor,\ycursorK);
%        \draw[dashed,Plasma33] (\xcursor,\ycursorK) -- (0,\ycursorK) node[left]{\num{1.4452e-4}};
%        \draw[dashed,Plasma66] (\xcursor,\ycursorV) -- (0,\ycursorV) node[left]{\num{9.1120e-5}};
%         \draw[dashed,black!50] ({axis cs:\xcursor,0}|-{rel axis cs:0,0}) -- ({axis cs:\xcursor,0}|-{rel axis cs:0,\ycursorK});
       \addplot[loosely dashed,draw=black, ultra thick] {\rh};
       \draw(0,\rh) node[above right]{\qty{\rh}{\meter}};
        
        \legend{$\delta_\kappa$ \\ $\delta_\nu$ \\ $r_h$ \\};
    \end{axis}
\end{tikzpicture}
    \caption{\'Evolution des épaisseurs de couches limites thermique et visqueuse en fonction de la fréquence, définies par le système d'équations~\eqref{eq:CouchesLimites}.}
    \label{fig:dKdV}
\end{figure}



\subsection{Chaîne d'excitation et d'acquisition}
L'instrumentation utilisée est basée sur celle conçue au début du projet \cite{ramadan_design_2021}, tout en modifiant quelques éléments.\bigskip

Premièrement, la chaîne d'excitation est présentée. Elle est assez simple, et se compose d'un générateur de fonction à deux canaux (TekTronix AFG3022). Chaque canal est ensuite connecté à un amplificateur pour chaque source acoustique. La source principale (RIX Industries, 1S241M) est alimentée par un amplificateur QSC PLD4.5, et la source secondaire (Peerless, GBS135F) par un amplificateur Yamaha P3500S.\medskip

%Un générateur basse fréquence génère les signaux d'alimentation des sources acoustiques. Il dispose de deux sorties, chacune connectée à un amplificateur avant d'être reliée aux transducteurs. Dans le cas de la source principale (RIX Industries, 1S241M), l'amplificateur est un QSC PLD4.5 tandis que pour la source secondaire (Peerless, GBS135F) il s'agit d'un Yamaha P3500S.

Ensuite, la chaîne d'acquisition se compose de plus de trente capteurs. Tous ne sont pas utilisés, mais peuvent servir de contrôle durant une expérience, pour s'assurer du bon déroulement de celle-ci.

\paragraph*{Alimentation électrique des sources} L'alimentation électrique de la source acoustique principale est mesurée au moyen d'une sonde différentielle pour la tension \echaf{et le courant ?}. Pour la source acoustique secondaire, un multimètre et une pince de courant se chargent de mesurer sa consommation électrique. En parallèle, les tensions aux bornes des deux sources sont affichées sur un oscilloscope pour s'assurer du déphasage.

\paragraph*{Température} Dix-neuf thermocouples Type K de \qty{.5}{\milli\meter} de diamètre sont placés de la manière suivante : quinze thermocouples mesurent la température en différentes positions du noyau, un devant la source acoustique principale, deux derrière celle-ci, et un derrière la source acoustique secondaire. La carte d'acquisition utilisée (National Instruments, NI9213) ne comporte que seize entrées, et pour toutes les mesures ce sont les thermocouples du noyau et de devant la source acoustique principale qui y sont connectés. Le placement de ces thermocouples d'intérêt est représenté sur la figure~\ref{fig:TCdansNoyau} par les symboles `\textcolor{cyan}{\textbullet}'.

\begin{figure}[!ht]
    \centering
    \external{fig_TCdansNoyau}
    %\externalremake
    \begin{tikzpicture}[scale=.2]
	\def\rCHX{14cm};
	\def\lCHX{.7cm};
	\def\rREG{14.8cm};
	\def\lREG{3.9cm};
	\def\rAHX{11cm};
	\def\lAHX{2.3cm};
	
	
	\fill[pattern=horizontal lines,pattern color=MatlabOrange,draw=black] (0,-\rAHX) rectangle ++(\lAHX,2*\rAHX);
	\draw[MatlabOrange] (0,-\rAHX) node(AHX)[below left]{\'Echangeur ambiant};
	\foreach \r in {-.9,0,.9}{
		\draw[cyan] (-.1*\lREG,\r*\rAHX) node{\textbullet};
	}
	\filldraw[draw=black,fill=gray!50!white] (0,\rAHX) rectangle (\lAHX,\rREG);		% côtés où l'eau circule
	\filldraw[draw=black,fill=gray!50!white] (0,-\rAHX) rectangle (\lAHX,-\rREG);	%
	
	\draw[MatlabOrange,->] (AHX.north) to[out=90,in=180] (-.1*\lCHX,-.5*\rAHX);
	
	\begin{scope}[xshift=\lAHX] % Reg
		\fill[pattern=crosshatch,pattern color=gray,draw=black] (0,-\rREG) rectangle ++(\lREG,2*\rREG);
		\draw[black] (\lREG/2,\rREG) node[above]{Régénérateur};		
		\foreach \x in {.1,.5,.9}{
			\foreach \r in {-.9,0,.9}{
				\draw[cyan] (\x*\lREG,\r*\rREG) node{\textbullet};
		}}
	\end{scope}
	
	\begin{scope}[xshift=\lAHX+\lREG] % CHX
		\fill[pattern=horizontal lines,pattern color=MatlabBlue,draw=black] (0,-\rCHX) rectangle ++(\lCHX,2*\rCHX);
		\draw[MatlabBlue] (\lCHX,-\rCHX) node(CHX)[below right]{\'Echangeur froid};
		\foreach \r in {-.9,0,.9}{
		\draw[cyan] (\lCHX+.1*\lREG,\r*\rCHX) node{\textbullet};
	}
	\filldraw[draw=black,fill=gray!50!white] (0,\rREG) rectangle (\lCHX,\rCHX);		% côtés où l'eau circule
	\filldraw[draw=black,fill=gray!50!white] (0,-\rREG) rectangle (\lCHX,-\rCHX);	%
	
	\draw[MatlabBlue,->] (CHX.north) to[out=90,in=0] (1.1*\lCHX,-.5*\rCHX);
	\end{scope}
	
	\draw[green!50!black] (0,0) node[left]{\begin{tabular}{rl}Source & \\ acoustique & $\leftarrow$ \\ secondaire &\end{tabular}};
	\draw[green!50!black] ({\lCHX+\lREG+\lAHX},0) node[right]{\begin{tabular}{rl}	
	 & Source \\ $\rightarrow$ \textcolor{cyan}{\textbullet} & acoustique \\ & principale\end{tabular}};
	
\end{tikzpicture}
    \caption{Emplacement des thermocouples dans le noyau thermoacoustique. Zoom sur l'encadré orange de la figure~\ref{fig:SchemaGeneralTACOT}.}
    \label{fig:TCdansNoyau}
\end{figure}

\paragraph*{Pression dynamique} Quatre sondes piézoélectriques (PCB Piezotronics, 113B28) captent les oscillations de pression dans la pompe à chaleur. Deux sont placées à l'arrière de chacune des sources acoustiques, et les deux autres dans le canal de rétroaction de la cavité thermoacoustique. Les capteurs sont ensuite connectés à une carte d'acquisition (National Instruments, NI9234). Ces sondes doivent supporter la pression statique élevée à l'intérieur de la machine, ainsi que l'amplitude acoustique nécessaire au processus thermoacoustique. De plus, elles doivent être affleurantes aux parois auxquelles elles sont montées, ce qui empêche leur installation au sein du noyau. Toutefois, la longueur d'onde dans le mélange de gaz vaut $\lambda=\qty{11.7}{\meter}$ à la fréquence de fonctionnement $f=\qty{47}{\hertz}$ et est suffisamment grande pour garantir une amplitude de pression constante dans toute la machine.

\paragraph*{Pression statique} Deux capteurs (Endress, Cerabar PMP21) sont connectés sur les deux tuyaux d'alimentation en gaz de la pompe  à chaleur d'un côté, et sur une carte d'acquisition (National Instruments, NI9234) de l'autre. Les arrivées de gaz se trouvent de part et d'autre de la source acoustique principale et on pour but d'éviter une surpression sur sa face avant ou arrière et son endommagement.

\paragraph*{Puissance extraite par l'échangeur ambiant} Le fonctionnement de cet échangeur est détaillé dans l'annexe~\ref{chap:AHX}. Pour déterminer la quantité de chaleur extraite du côté ambiant du noyau, la différence de température entre l'entrée d'eau de l'échangeur et sa sortie d'eau est mesurée grâce à deux sondes de platine PT100 connectées sur une carte d'acquisition (National Instruments, NI9217).

\paragraph*{Déplacement des sources} Le piston de chaque source acoustique est équipé d'un accéléromètre. Pour la source acoustique principale, l'accéléromètre (MMF, KS91C) est collé sur la face arrière, tandis que pour la source secondaire, le capteur (PCB Piezotronics, 352C23) est collé sur la face avant. Ces capteurs sont choisis de sorte à ne pas trop varier la masse de l'équipage mobile, en particulier pour la source secondaire où la masse du piston et celle de l'ensemble accéléromètre et câble sont du même ordre de grandeur.\bigskip

Toutes les connexions entre l'intérieur de la machine sous haute pression statique et l'extérieur se font via des traversées étanches. Pour les capteurs, il s'agit de HF2-8CU+16K de Spectite, dimensionnées pour \qty{550}{\bar}. Pour les sources acoustiques, une traversée FA17613 de Solid Sealing Technology est choisie, et pour la source acoustique secondaire, le modèle FA36735 du même fabricant est retenu.

%Les signaux de tensions aux bornes des sources acoustiques sont acquis par une carte d'acquisition (National Instruments, (\echaf{modèle}), après connexion à une sonde de tension (\echaf{modèle}). Deux accéléromètres (\echaf{modèle}) sont collés sur les sources pour mesurer leur déplacement. Pour connaître la pression acoustique dans la cavité thermoacoustique, quatre sondes (\echaf{modèle}) sont placées respectivement à l'arrière de la source principale, à l'arrière de la source secondaire 


%\begin{figure}[!ht]
%    \centering
%    \external{fig_ChaineAcqui}
%    %\externalremake
%    \input{../fig/fig_ChaineAcqui/tex/fig_ChaineAcqui.tex}
%    \caption{Carte de la chaine d'acquisition et d'alimentation du réfrigérateur TACOT}
%    \label{fig:ChaineAcqui}
%\end{figure}

%\begin{itemize}
%    \item GBF
%    \item Amplis
%    \begin{itemize}
%        \item QSC
%        \item Yamaha
%    \end{itemize}
%    \item Sondes de tension
%    \item Cartes NI
%    \begin{itemize}
%        \item Pression statique
%        \item Pression dynamique
%        \item Thermocouples
%        \item PT100
%        \item Accéléromètres
%    \end{itemize}
%    \item LabVIEW d'acquisition
%\end{itemize}

%Pour étudier la distribution de température le long de l'axe du noyau, ainsi que dans les dimensions transverses, Seize thermocouples sont placés sur un plan et représentés par les symboles~`\textcolor{cyan}{\textbullet}' sur la figure~\ref{fig:TCdansNoyau}. Neuf sont placés au c\oe{}ur du noyau, dans le régénérateur. Trois sont fixés à l'extérieur du noyau, hors de l'échangeur ambiant, et trois autres sur l'extérieur de l'échangeur froid. Enfin, un dernier thermocouple est positionné au voisinage de la source acoustique principale, en vis-à-vis de l'échangeur froid.



%\subsection{Emplacement des capteurs}
%
%\begin{figure}[!ht]
%    \centering
%    \external{fig_ThermocouplesDefinition}
%    %\externalremake
%    \begin{tikzpicture}
    \def\LX{1};
    \def\LY{2};
    \def\CoreX{1.5};
    \def\CoreY{.9*\LY};
    
    \draw[line width=.5mm] (-2.5*\LX,0) to[out=90,in=-180] (-\LX,\LY) -- ++(2*\LX,0) -- ++(.5*\LX,-2*\LY/3) -- ++(.2*\LX,0) -- ++(0,2*\LY/3);
\draw[line width=.5mm] (\LX,\LY) -- ++(\LX,0) to[out=0,in=90] (3.5*\LX,0);

\draw[line width=.5mm] (-\LX,\CoreY) -- ++(\CoreX,0);
\draw[fill=PythonBlue] (-.9*\LX,0) -- ++(0,\CoreY) to[out=-80,in=90] (-.7*\LX,0);
\draw ({-\LX+.4*\CoreX},0) -- ++(0,\CoreY);
\draw ({-\LX+.9*\CoreX},0) -- ++(0,\CoreY);

\draw[fill=PythonBlue] (1.6*\LX,0) |- ++(.3*\LX,.9*\LY/3) |- ++(\LX,.2*\LY) arc (90:0:.05) -- ++(0,-.5*\LY);
    
    \begin{scope}[xscale=1,yscale=-1]
        \draw[line width=.5mm] (-2.5*\LX,0) to[out=90,in=-180] (-\LX,\LY) -- ++(2*\LX,0) -- ++(.5*\LX,-2*\LY/3) -- ++(.2*\LX,0) -- ++(0,2*\LY/3);
\draw[line width=.5mm] (\LX,\LY) -- ++(\LX,0) to[out=0,in=90] (3.5*\LX,0);

\draw[line width=.5mm] (-\LX,\CoreY) -- ++(\CoreX,0);
\draw[fill=PythonBlue] (-.9*\LX,0) -- ++(0,\CoreY) to[out=-80,in=90] (-.7*\LX,0);
\draw ({-\LX+.4*\CoreX},0) -- ++(0,\CoreY);
\draw ({-\LX+.9*\CoreX},0) -- ++(0,\CoreY);

\draw[fill=PythonBlue] (1.6*\LX,0) |- ++(.3*\LX,.9*\LY/3) |- ++(\LX,.2*\LY) arc (90:0:.05) -- ++(0,-.5*\LY);
    \end{scope}
    
    
    \draw[dashed,rounded corners,PythonRed] (.55,.95*\LY) rectangle ++(-.75*\CoreX,-1.9*\LY) node[midway]{\rotatebox{90}{Noyau TA}};
    
    \begin{scope}[xshift=5cm,xscale=2.5,yscale=2]
        \draw (0,0) |- ++(2,1.5) -- ++(0,-1.5);
        \draw (.5,0) -- ++(0,1.5);
        \draw (1.5,0) -- ++(0,1.5);
        \draw[line width=1mm] (-.2,1.5) -- ++(2.4,0);
    
        \begin{scope}[xscale=1,yscale=-1]
            \draw (0,0) |- ++(2,1.5) -- ++(0,-1.5);
            \draw (.5,0) -- ++(0,1.5);
            \draw (1.5,0) -- ++(0,1.5);
            \draw[line width=1mm] (-.2,1.5) -- ++(2.4,0);
        \end{scope}
        % \foreach \x [evaluate=\x] in {0,...,4}{
        %     \foreach \y [evaluate=\y] in {1,...,3}{
        %     \draw (\x,\y) node[]{$t$};}}
    \end{scope}
    
    %\draw[line width=1mm] (5*\LX,.5*\LY) -- ++(6*\LX,0);
    %\draw[line width=1mm] (5*\LX,-.5*\LY) -- ++(6*\LX,0);
    %
    %\draw (-2.5*\LX,\LY) node[above]{\bf (a)};
    %\draw (5*\LX,\LY) node[above]{\bf (b)};
    
\end{tikzpicture}
%    \caption{Emplacements des thermocouples dans le noyau thermoacoustique}
%    \label{fig:ThermocouplesDefinition}
%\end{figure}

\section{Protocole expérimental}\label{chap:ProtocolExpe}
Pour l'étude de l'influence de la gravité sur la distribution de température dans son noyau et ses performances, le réfrigérateur doit pouvoir être orienté dans toutes les orientations utiles. Pour ce faire, il est suspendu par des palans grâce aux fixations situées à ses extrémité  et au milieu dans le sens de sa longueur. La figure~\ref{fig:TACOTSuspendu_Frigo} présente le réfrigérateur accroché à ses extrémités, et la figure~\ref{fig:TACOTSuspendu_Palans} les trois palans pour le soutenir. Les deux palans de couleur grise, initialement présents pour régler l'inclinaison de la pompe à chaleur par rapport à l'axe horizontal, et le troisième de couleur bleue pour ajouter une direction de rotation autour de l'axe de symétrie. Celui-ci permet en outre de plus aisément passer d'une orientation à l'autre. 

\begin{figure}[!ht]
    \centering
	\begin{subfigure}{.47\textwidth}
		\centering
		\includegraphics[width=\textwidth]{../fig/fig_SystemeAccroche/Machine_horizBetter_cropped.jpg}
		\caption{}
		\label{fig:TACOTSuspendu_Frigo}
	\end{subfigure}		%
	\begin{subfigure}{.47\textwidth}
		\centering
		\includegraphics[width=\textwidth]{../fig/fig_SystemeAccroche/Palans.jpg}
		\caption{}
		\label{fig:TACOTSuspendu_Palans}
	\end{subfigure}	    
    \caption{Photographies \subref{fig:TACOTSuspendu_Frigo} du refrigérateur accroché et \subref{fig:TACOTSuspendu_Palans} des palans formant le système de suspension.}
    \label{fig:TACOTSuspendu}
\end{figure}

\subsection{Définition des orientations}

Les orientations choisies au moyen des palans sont décrites par deux angles $\psi_v$ et $\psi_h$. Le premier désigne l'angle entre l'axe horizontal et l'axe de symétrie du réfrigérateur, tandis que le second, la rotation autour de cet axe de symétrie. Les orientations utilisées dans les différentes parties de ce manuscript sont présentées sur la figure~\ref{fig:OrientationCore}. Cette figure, dans laquelle la gravité est toujours dirigée vers le bas de la page, présente également les emplacements et les numéros d'identification des thermocouples utilisés.\medskip

\begin{figure}[!ht]
    \centering
	\begin{subfigure}[c]{.47\textwidth}
		\centering
		\begin{tikzpicture}[scale=2/3]

%    \def\lenreg{2};
%    \def\diam{3};
    \def\spy{2};
    \def\xdist{8cm};
    \def\ydist{-7cm};
%    \def\persp{20};
%    
%    \def\LX{1};
%    \def\LY{2};
%    \def\CoreX{1.5};
%    \def\CoreY{.9*\LY};
%    

	\def\L{2.1};
	\def\R{5};
	\def\HX{.25};
	\def\decalage{\R/2-\L/2};
	
		
			
		\fill[right color=MatlabBlue,left color=MatlabOrange, draw=black] (\decalage,0) rectangle ++(\L,\R);
		\draw[fill=MatlabOrange] (\decalage,0) rectangle ++(-\HX,\R);
		\draw[fill=MatlabBlue] (\decalage+\L,0) rectangle ++(\HX,\R);

		\foreach \z [evaluate=\z] in {0,...,4}{
			\foreach \r [evaluate=\r as \num using int(\r+1 + 3*\z)] in {0,...,2}{
				\draw ({\decalage+.5+\L-\z*(1+\L)/4},{-(\R-.4)/2*\r+\R-.2}) node[minimum size=10pt,draw,circle,fill=white,opacity=.7,text opacity=1]{} node(n\z\r){\scriptsize \num};
}}

%		\draw (n01.east) node [right]{0 $\rightarrow$ \begin{tabular}{l}Source\\acoustique\\principale\end{tabular}};
		\draw ($(n01)+(1.5,0)$) node[minimum size=10pt,draw,circle,fill=white,opacity=.7,text opacity=1]{} node(RIX) {\scriptsize 0};% node[anchor=west]{\begin{tabular}{rl}
%		& Source\\
%		$\rightarrow$ & acoustique\\
%		& principale
%		\end{tabular}};
		\draw (n30.north west) node [above, fill=white, fill opacity=.7, text opacity=1]{\textcolor{MatlabOrange}{\textbf{Ambiant}}};
		\draw (n10.north east) node [above, fill=white, fill opacity=.7, text opacity=1]{\textcolor{MatlabBlue}{\textbf{Froid}}};
\end{tikzpicture}
		\caption{`\texttt{H1}'}
		\label{fig:OrientationCore_H1}
	\end{subfigure}		%
	\begin{subfigure}[c]{.47\textwidth}
		\centering
		\begin{tikzpicture}[scale=2/3]

%    \def\lenreg{2};
%    \def\diam{3};
    \def\spy{2};
    \def\xdist{8cm};
    \def\ydist{-7cm};
%    \def\persp{20};
%    
%    \def\LX{1};
%    \def\LY{2};
%    \def\CoreX{1.5};
%    \def\CoreY{.9*\LY};
%    

	\def\L{2.1};
	\def\R{5};
	\def\HX{.25};
	\def\decalage{\R/2-\L/2};
	
%	\draw[opacity=0] (\decalage,0) rectangle ++(-\HX,\R); %%% Pour l'alignement vertical
	
	\begin{scope}[xslant=1,yscale=.5]
		
		\fill[shading=axis,right color=MatlabBlue,left color=MatlabOrange, shading angle=45, draw=black] (\decalage,0) rectangle ++(\L,\R);
		\draw[fill=MatlabOrange] (\decalage,0) rectangle ++(-\HX,\R);
		\draw[fill=MatlabBlue] (\decalage+\L,0) rectangle ++(\HX,\R);

		\foreach \z [evaluate=\z] in {0,...,4}{
			\foreach \r [evaluate=\r as \num using int(\r+1 + 3*\z)] in {0,...,2}{
				\draw ({\decalage+.5+\L-\z*(1+\L)/4},{-(\R-.4)/2*\r+\R-.2}) node[minimum size=10pt,draw,circle,fill=white,opacity=.7,text opacity=1]{} node(n\z\r){\scriptsize \num};
}}

%		\draw (n01.east) node [right]{$\rightarrow$ \begin{tabular}{l}Source\\acoustique\\principale\end{tabular}};
		\draw ($(n01)+(1.5,0)$) node[minimum size=10pt,draw,circle,fill=white,opacity=.7,text opacity=1]{} node {\scriptsize 0};% node[anchor=west]{\begin{tabular}{rl}
%		& Src\\
%		$\rightarrow$ & ac\\
%		& princ
%		\end{tabular}};
		\draw (n30.north west) node [above, fill=white, fill opacity=.7, text opacity=1]{\textcolor{MatlabOrange}{\textbf{Ambiant}}};
		\draw (n10.north east) node [above, fill=white, fill opacity=.7, text opacity=1]{\textcolor{MatlabBlue}{\textbf{Froid}}};
	\end{scope}
	
	
		
\end{tikzpicture}
		\caption{`\texttt{H2}'}
		\label{fig:OrientationCore_H2}
	\end{subfigure} \\ \vspace{1cm}
	\begin{subfigure}[c]{.47\textwidth}
		\centering
		%\fill[top color=red!25, bottom color=blue!25, draw=black] (0,0) rectangle ++(\R,\L);
%\draw[fill=blue!25] (0,0) rectangle ++(\R,-\HX);
%\draw[fill=red!25] (0,\L) rectangle ++(\R,\HX);
%
%\foreach \z [evaluate=\z] in {0,...,4}{
%	\foreach \r [evaluate=\r as \num using int(\r+1 + 3*\z)] in {0,...,2}{
%		\draw ({-(\R-.4)/2*\r+\R-.2},{\z*(1+\L)/4-.5}) node(n\z\r){\num};
%}}
%
%\draw (n40.south east) node [right]{AHX};
%\draw (n00.north east) node[right]{CHX};
%\draw (n01.south) node [below]{\shortstack{ $\downarrow$ \\Source acoustique principale}};
%
%\draw (0,\L+2*\HX+\spy) node [anchor=west]{\textbf{(c)} \texttt{V1}};

\begin{tikzpicture}[scale=2/3]

%    \def\lenreg{2};
%    \def\diam{3};
    \def\spy{2};
    \def\xdist{8cm};
    \def\ydist{-7cm};
%    \def\persp{20};
%    
%    \def\LX{1};
%    \def\LY{2};
%    \def\CoreX{1.5};
%    \def\CoreY{.9*\LY};
%    

	\def\L{2};
	\def\R{5};
	\def\HX{.35};
	\def\decalage{\R/2-\L/2};
	
	\begin{scope}[yslant=tan(22.5)]	
		
		\node at (0,-\HX) (NewO) {};
	
		\fill[shading=axis,right color=MatlabBlue,left color=MatlabOrange, shading angle=22.5, draw=black] (0,0) rectangle ++(\R,\L);
		\draw[fill=MatlabBlue] (0,0) rectangle ++(\R,-\HX);
		\draw[fill=MatlabOrange] (0,\L) rectangle ++(\R,\HX);

		\foreach \z [evaluate=\z] in {0,...,4}{
			\foreach \r [evaluate=\r as \num using int(\r+1 + 3*\z)] in {0,...,2}{
				\draw ({-(\R-.4)/2*\r+\R-.2},{\z*(1+\L)/4-.5}) node[minimum size=10pt,draw,circle,fill=white,opacity=.7,text opacity=1]{} node(n\z\r){\scriptsize \num};
}}

%		\draw (n40.south east) node [right, fill=white, fill opacity=0, text opacity=1]{\textcolor{MatlabOrange}{\textbf{Ambiant}}};
%		\draw (n00.north east) node[right, fill=white, fill opacity=0, text opacity=1]{\textcolor{MatlabBlue}{\textbf{Froid}}};
		\draw ($(n01.south)+(0,-1.1)$) node[minimum size=10pt,draw,circle,fill=white,opacity=.7,text opacity=1]{} node (RIX){\scriptsize 0};% node[anchor=north]{\begin{tabular}{c}
%		$\downarrow$\\
%		Source acoustique principale
%		\end{tabular}};

	\end{scope}
	\begin{pgfonlayer}{background}
		\draw[->, very thick] (NewO.center) -- ++(22.5:1.2*\R) node [above] {$\mathbf e_{y,0}$};
		\draw[->, very thick] (NewO.center) -- ++(90:1.2*\R) node [left] {$\mathbf e_{z,0}$};
		\draw[->, very thick] (NewO.center) -- ++(-45:1.2*\R) node [right] {$\mathbf e_{x,0}$};
  	\end{pgfonlayer}		
\end{tikzpicture}
		\caption{`\texttt{V1}'}
		\label{fig:OrientationCore_V1}
	\end{subfigure} %
	\begin{subfigure}[c]{.47\textwidth}
		\centering
		%\fill[top color=blue!25, bottom color=red!25, draw=black] (0,0) rectangle ++(\R,\L);
%\draw[fill=red!25] (0,0) rectangle ++(\R,-\HX);
%\draw[fill=blue!25] (0,\L) rectangle ++(\R,\HX);
%
%\foreach \z [evaluate=\z] in {0,...,4}{
%	\foreach \r [evaluate=\r as \num using int(\r+1 + 3*\z)] in {0,...,2}{
%		\draw ({(\R-.4)/2*\r+.2},{-\z*(1+\L)/4+\L+.5}) node(n\z\r){\num};
%}}
%
%\draw (n01.north) node [above]{\shortstack{Source acoustique principale\\ $\uparrow$}};
%\draw (n42.north east) node [right]{AHX};
%\draw (n02.south east) node [right]{CHX};
%
%\draw (0,\L+2*\HX+\spy) node [anchor=west]{\textbf{(d)} \texttt{V2}};

\begin{tikzpicture}[scale=2/3]

%    \def\lenreg{2};
%    \def\diam{3};
    \def\spy{2};
    \def\xdist{8cm};
    \def\ydist{-7cm};
%    \def\persp{20};
%    
%    \def\LX{1};
%    \def\LY{2};
%    \def\CoreX{1.5};
%    \def\CoreY{.9*\LY};
%    

	\def\L{1.5};
	\def\R{5};
	\def\HX{.25};
	\def\decalage{\R/2-\L/2};

		\fill[top color=blue!25, bottom color=red!25, draw=black] (0,0) rectangle ++(\R,\L);
		\draw[fill=red!25] (0,0) rectangle ++(\R,-\HX);
		\draw[fill=blue!25] (0,\L) rectangle ++(\R,\HX);

		\foreach \z [evaluate=\z] in {0,...,4}{
			\foreach \r [evaluate=\r as \num using int(\r+1 + 3*\z)] in {0,...,2}{
				\draw ({(\R-.4)/2*\r+.2},{-\z*(1+\L)/4+\L+.5}) node[minimum size=10pt,draw,circle,fill=white,opacity=.7,text opacity=1]{} node(n\z\r){\scriptsize \num};
}}

		\draw ($(n01.north)+(0,.5cm)$) node[minimum size=10pt,draw,circle,fill=white,opacity=.7,text opacity=1]{} node(RIX){\scriptsize 0} node[anchor=south]{\begin{tabular}{c}
		Source acoustique principale\\
		$\uparrow$
		\end{tabular}};
		\draw (n42.north east) node [right]{Ambiant};
		\draw (n02.south east) node [right]{Froid};
%		\draw (n41.south) node [below]{\textcolor{white}{\shortstack{Source acoustique principale\\ $\uparrow$}}};
		
\end{tikzpicture}
		\caption{`\texttt{V2}'}
		\label{fig:OrientationCore_V2}
	\end{subfigure}   
    \caption{Différentes orientations du c\oe{}ur thermoacoustique avec les positions des thermocouples et leurs numéro. Pour chaque cas, la gravité est orientée vers le bas. Les orientations correspondent aux angles \subref{fig:OrientationCore_H1}~$\psi_v=\ang{0}$ et $\psi_h=\ang{0}$ pour l'orientation `\texttt{H1}', \subref{fig:OrientationCore_H2}~$\psi_v=\ang{0}$ et $\psi_h=\ang{+90}$ pour l'orientation `\texttt{H2}', \subref{fig:OrientationCore_V1}~$\psi_v=\ang{-90}$ pour l'orientation `\texttt{V1}', et \subref{fig:OrientationCore_V2}~$\psi_v=\ang{+90}$ pour l'orientation `\texttt{V2}'.}%\textcolor{red}{CHX et AHX OK ou éch. froid et éch. chaud ? + $\psi_i$ dans la caption ou la figure ?}}
    \label{fig:OrientationCore} %
\end{figure}



La première orientation, nommée `\texttt{H1}' et représentée sur la figure~\ref{fig:OrientationCore_H1}, est la même que dans l'article dédié à la conception du réfrigérateur \cite{ramadan_design_2021}. Dans cette configuration, le \textsc{Tacot} est placé à l'horizontale comme sur la figure~\ref{fig:TACOTSuspendu_Frigo}, et les thermocouples sont placés sur un plan vertical coplanaire à la gravité. Cette orientation fait office de référence des orientations, soit $\psi_v=\psi_h=\qty{0}{\degree}$.\smallskip

Ensuite, la deuxième orientation est représentée sur la figure~\ref{fig:OrientationCore_H2}. Dans ce cas, référérencé en tant que `\texttt{H2}', le réfrigérateur est toujours à l'horizontale ($\psi_v=\qty{0}{\degree}$), mais pivoté autour de son axe pour placer les thermocouples sur un plan horizontal auquel la gravité est orthogonale ($\psi_h=\qty{90}{\degree}$).\smallskip

L'orientation `\texttt{V1}' est affichée sur la figure~\ref{fig:OrientationCore_V1}. Cette configuration est radicalement différentes des deux précédentes : l'axe de symétrie du réfrigérateur est vertical, avec l'échangeur froid sous l'échangeur ambiant, soit $\psi_v=\qty{-90}{\degree}$. \smallskip

Enfin, l'orientation `\texttt{V2}' affichée sur la figure~\ref{fig:OrientationCore_V2} est l'orientation inverse de la précédente. L'axe de symétrie du réfrigérateur est encore vertical, mais la source acoustique principale est cette fois au dessus du noyau thermoacoustique et $\psi_v=\qty{+90}{\degree}$.

\subsection{Acquisitions}
Les acquisitions sont réalisées en plusieurs temps. Tout d'abord et pour toutes les expériences,  l'état initial de toutes les grandeurs est acquis sur une minute et sauvegardé sous un label `\texttt{init}' à chaque début de journée de campagne. Cela permet de garder en mémoire toutes les conditions expérimentales initiales dont les valeurs peuvent potentiellement influer sur le comportement du réfrigérateur, comme par exemple la température ambiante ou la pression statique. \medskip

Ensuite, en prévision de la mesure de flux de chaleur $\dot Q_a$ extrait par l'échangeur ambiant (voir l'annexe~\ref{chap:AHX}), l'eau est préalablement mise en circulation dans cet échangeur après avoir démarré une acquisition des 30 capteurs jusqu'à stabilisation de la distribution de température dans le noyau. L'acquisition est ensuite interrompue et enregistrée avec un label `\texttt{Water}'. \bigskip

L'étape suivante dépend du type d'expérience menée : les mesures peuvent être sans ou avec acoustique, et ce, pour  différentes amplitudes de pression oscillante. En revanche, certains des paramètres d'excitation restent constants pour toutes les expériences :  le gaz est également le même dans toutes les expériences. Il est composé de \qty{65}{\percent} d'hélium et de \qty{35}{\percent} d'argon, car dans ces proportions le nombre de Prandtl est minimum \cite{belcher_working_1999} ; ce mélange est ensuite pressurisé à \qty{40}{\bar}. Dans le cas des expériences avec acoustique, le modèle \textsc{DeltaEC} prédit les meilleurs performances à la fréquence $f=\qty{47}{\hertz}$, c'est-à-dire la fréquence de résonance du système. C'est par ailleurs le seul point de fonctionnement où l'impédance électrique est supérieure à la limite basse admise par l'amplificateur, soit \qty{2}{\ohm}. Ensuite, le déphasage inter-source $\varphi_{2-1}$ est également fixé à \ang{-60} pour toutes les expériences, également indiqué comme déphasage optimale par les simulations et que des expériences préliminaires confirment.

\subsubsection{Mesures sans acoustique}\label{chap:MesureSansAcou}
Pour ces mesures de type `\texttt{heat\_{}only}', la charge thermique est appliquée au noyau sans alimenter les sources acoustiques. Cette charge thermique consiste en l'alimentation électrique de cartouches chauffantes contenues dans l'échangeur froid par une puissance connue, tandis qu'un débit d'eau de \qty{7}{\litre\per\minute} s'écoule dans l'échangeur ambiant qui se trouve de l'autre côté du noyau. 

Ces mesures doivent permettre d'étudier la distribution de température en l'absence d'écoulement oscillant, ainsi que de calculer les valeurs de conductivité thermique $k_x$ et $k_r$ ou les coefficients de pertes latérales $h_x$ et $h_r$.\medskip

Dans ce type d'expériences, les noms des zones \og froide \fg{} et \og ambiante \fg{} sont conservés pour des raisons de cohérence avec les schémas présentés auparavant, mais l'eau circulant dans l'échangeur ambiant et les cartouches chauffantes se trouvant dans l'échangeur froid, la direction du gradient de température dans le noyau thermoacoustique est inversée par rapport aux expériences avec acoustique. %Toutefois, les ordres de grandeur des différences de température sont les mêmes que pour les expériences avec acoustique, c'est-à-dire 

%Pour garder un moyen de comparaison avec les mesures avec acoustique, le mélange de gaz est le même.

\subsubsection{Mesures avec acoustique}\label{chap:MesureAvecAcou} 
Une acquisition étiquetée `\texttt{Acou}' est démarrée, puis les sources sont alimentées jusqu'à l'amplitude souhaitée. Au bout d'une heure, l'acquisition est arrêtée et sauvegardée. En l'absence d'expérience avec charge thermique, c'est la fin de l'expérience : toutes les sources acoustiques et circulations d'eau sont progressivement arrêtées et le réfrigérateur est laissé pour un retour à l'état initial.

Au cours de cette étude, trois amplitudes acoustiques sont choisies. La première correspond à un \textit{drive ratio} $DR=\frac{p}{P_0}=\qty{.4}{\percent}$, soit une amplitude très faible où l'effet thermoacoustique est à peine visible -- soit un gradient de température de l'ordre de \qty{5}{\degreeCelsius}. Ainsi, l'hypothèse concernant la linéarité acoustique est mieux vérifiée et peut \textit{a priori} être plus aisément comparé à la théorie linéaire. À l'inverse, le \textit{drive ratio} de la deuxième amplitude est le plus élevé avec $DR=\qty{3.5}{\percent}$, et est celui pour lequel les performances du réfrigérateur ($COP$, $Q_f$, ...) sont les plus élevées obtenues avec cette machine \cite{ramadan_design_2021}, mais aussi qui présentent de forts écarts à la théorie. La troisième est choisie à un \textit{drive ratio} intermédiaire où $DR=\qty{2}{\percent}$. 

Ensuite, une charge thermique peut être appliquée au noyau, par le biais de l'échangeur froid. Il contient six cartouches chauffantes connectées en parallel et alimentées électriquement par un transformateur. Pour une expérience donnée, une puissance thermique est choisie selon la relation

\begin{equation}
	Q_f = \frac{E^2}{R},
	\label{eq:Qf_définitionEsurR}
\end{equation}
où $E$ est la tension appliquée aux cartouches et $R=\qty{22.4}{\ohm}$ la résistance des cartouches en parallel. \echaf{à continuer}



\begin{comment}
\subsubsection{Paramètres d'acquisition}
Comme dit précédemment, la fréquence de fonctionnement du \textsc{Tacot} est de \qty{47}{\hertz}, ce qui implique une fréquence d'échantillonnage au moins deux fois supérieure. Cependant, les cartes d'acquisition sont rassemblées sur une baie d'instrumentation, contraignent la fréquence d'échantillonage utilisée. Celles concernant les mesures de quantité oscillantes (pression acoustique, accélération \echaf{à vérifier}) imposent que la fréquence d'échantillonage $f_s$ soit au moins égale à \qty{1651}{\Hz}\footnote{Les acquisitions des \num{30} capteurs durent \qty{1}{\hour}, et les données sont encodées sur \qty{32}{\bit} flottants. Au total, chaque acquisition pèse \qty{713}{\mega\byte}, taille à laquelle il faut ajouter quelques \unit{\mega\byte} pour le protocole \texttt{tdms} et l'entête contenant les informations de mesure.}.
\end{comment}

\section{Ensemble des simulations réalisées}\label{chap:SimusRealisees}
\subsection{Estimation théorique du flux de chaleur de convection naturelle}
Au sein du réfrigérateur \textsc{Tacot} et particulièrement dans la cavité devant la source acoustique principale, la distribution de température du côté froid hors du noyau laisse penser à la présence d'une cellule de convection naturelle à l'intérieur. Il est difficile de se rendre compte des flux massique et thermique causés par la différence de température de part et d'autre des différentes zones du \textsc{Tacot} -- volume d'adaptation d'impédance, noyau thermoacoustique -- à cause de leurs géometries, du type de convection naturelle rencontré, de la porosité, etc. Des études hydrodynamiques sont menées pour aider à l'interprétation des mesures de température. \medskip

Tout d'abord, deux étude très simplifiées sont réalisées pour une cavité 2D différentiellement chauffée par des températures chaude $T_c$ et froide $T_f$. Ces études doivent permettre l'obtention d'ordres de grandeurs des quantité d'intérêt, en particulier le flux de chaleur $Q_{conv}$ qui agit comme une charge thermique sur le côté froid du noyau thermoacoustique. \smallskip

Ensuite, des simulations par éléments finis de cette cavité et sur le régénérateur sur le logiciel Comsol Multiphysics permettent d'estimer les lignes de courants dans la cellule et l'influence de cet écoulement sur la distribution de température sur l'échangeur froid, en plus de déterminer des paramètres clés pour la compréhension des phénomènes thermiques en jeu.

\subsubsection{\'Etudes simplifiées}
\paragraph{Sans acoustique}
Pour introduire des concepts utiles à la compréhension des phénomènes de convection naturelle, une étude très simplifiée dans une cavité rectangulaire en 2D et représentée sur les figures~\ref{fig:SimuConvNat2D}{\color{MatlabOrange}(a)} et {\color{MatlabOrange}(b)} est menée. 

Dans la première sous-figure~\ref{fig:SimuConvNat2D}{\color{MatlabOrange}(a)}, les parois verticales droite et gauche sont respectivement maintenues à une température froide $T_f$ et chaude $T_c$, tandis que le sol, le plafond et le gaz au repos sont à la température $T_\infty$. En régime stationnaire, il s'établit une cellule de convection naturelle dans laquelle le gaz est mis en mouvement par les  variations de masse volumique proches des parois verticales. Cette configuration s'apparente aux orientations `\texttt{H1}' et `\texttt{H2}', respectivement présentées sur les figures~\ref{fig:OrientationCore_H1} et \subref{fig:OrientationCore_H2}.

Dans la seconde sous-figure~\ref{fig:SimuConvNat2D}{\color{MatlabOrange}(b)}, ce sont cette fois les sol et plafond qui sont fixés aux températures chaude $T_c$ et froide $T_f$, et les murs et le gaz au repos pour lesquels la température est $T_\infty$. Dans cette configuration, favorable a priori à la mise en place d'une instabilité de \og Rayleigh-Bénard \fg{}, il peut s'établir des cellules de convection naturelle de forme plus ou moins complexe au delà du nombre de Rayleigh critique $\Rayleigh_c$ compris entre 650 et 1700 pour des parois à température fixe \cite{getling_rayleigh-benard_1998}. Le gaz s'élève depuis la paroi chaude jusqu'à la paroi froide, de laquelle il redescend ensuite pour revenir à son point de départ. Dans ce cas, la cellule de convection naturelle peut adopter une structure très complexe, plus que ce que peut suggérer la figure~\ref{fig:SimuConvNat2D}{\color{MatlabOrange}(b)} qui ne représente qu'une illustration grossière du mouvement du fluide. Les expériences correspondant à ce cas sont mises en place en suivant les orientations `\texttt{V1}' et `\texttt{V2}', présentés respectivement sur les figures~\ref{fig:OrientationCore_V1} et \subref{fig:OrientationCore_V2}

\begin{figure}[!ht]
    \centering
    \external{fig_SimuConvNat2D}
%    \externalremake
    \input{../fig/fig_SimuConvNat2D/tex/fig_SimuConvNat2D}
    \caption{Cellule de convection naturelle dans une cavité rectangulaire 2D \textbf{(a)} pour un gradient de température normal à la direction de la gravité, et \textbf{(b)} pour un gradient de température colinéaire à la direction de la gravité. Quelque soit la configuration, la distance caractéristique est mesurée dans la direction verticale.}
    \label{fig:SimuConvNat2D}
\end{figure}



Il est possible de modéliser l'écoulement dans ce volume en utilisant les équations de Navier-Stokes avec l'approximation de Boussinesq qui s'écrivent

\begin{subequations}
	\begin{align}
		\partial_t \delta\rho + (\mathbf v \cdot \nabla)\mathbf{\delta\rho} + \delta\rho \nabla \cdot \mathbf{v} &= 0, \label{eq:NavierStokes_Boussinesq_Conti}\\
		\delta\rho [\partial_t \mathbf v + (\mathbf v \cdot \nabla)\mathbf v] &= -\nabla p + \mu \nabla^2 \mathbf v + \delta\rho \mathbf g, \text{et}\label{eq:NavierStokes_Boussinesq_QtMvt}\\
		\partial_t T + (\mathbf v \cdot \nabla) T - \kappa\nabla^2T &= 0. \label{eq:NavierStokes_Boussinesq_NRJinterne}
	\end{align}
	\label{eq:NavierStokes_Boussinesq}%
\end{subequations}
La dédimensionalisation de l'équation \eqref{eq:NavierStokes_Boussinesq_QtMvt} qui concerne la conservation de la quantité de mouvement donne

\begin{equation}
	\frac{1}{\mathrm{Pr}}(\partial_t \mathbf{v} + (\mathbf{v} \cdot \nabla)\mathbf{v}) = -\nabla p + \mathrm{Ra} T \mathbf e_z + \nabla^2 \cdot \mathbf{v},
	\label{eq:NonDim_NavierStokes_Boussinesq_QtMvt}
\end{equation}
et fait apparaître le nombre de Prandtl noté $\mathrm{Pr}$ déjà présenté dans l'équation \eqref{eq:Prandtl}, ainsi que le nombre de Rayleigh noté $\mathrm{Ra}$, et dont la définition est donnée par 
\begin{equation}
	\mathrm{Ra} = \frac{g \beta L_c^3}{\nu \kappa} (T_c-T_f),
	\label{eq:NbrRayleigh}
\end{equation}
où $T_c$ et $T_f$ sont les températures chaude et froide de part et d'autre de la zone considérée, et $L_c$ est la dimension caractéristique de la cavité suivant la direction verticale. Ce nombre est primordial car il correspond au rapport des effets gravifiques qui mettent le fluide en mouvement aux effets qui le limitent, soit la diffusion thermique qui limite la différence de température et la viscosité qui ralentit l'écoulement du fluide. Sa valeur indique également le régime de l'écoulement causé par la convection car des vitesses de référence verticales et horizontales, notées $v_{ref}^{// \mathbf g}$ et $v_{ref}^{\perp \mathbf g}$, peuvent d'ailleurs être calculées en fonction de ce nombre de Rayleigh suivant les définitions

\begin{subequations}
	\begin{align}
		v_{\sf ref}^{// \mathbf g} &\sim \frac{\kappa}{L_c}\sqrt{\Rayleigh} \text{ et}	\label{eq:VitesseReferenceV_Rayleigh}\\
		v_{\sf ref}^{\perp \mathbf g} &\sim \frac{\kappa}{L_c}\sqrt[4]{\Rayleigh},	\label{eq:VitesseReferenceH_Rayleigh}
	\end{align}
	\label{eq:VitesseReference_Rayleigh}%
\end{subequations}
d'après la réécriture en 2D des équations de conservation de la quantité de mouvement et de l'énergie par Belleoud \cite{belleoud_etude_2016} et dans le cas où le gradient de température est horizontal.\medskip

Dans un matériau poreux, il peut également exister des écoulements liés à la convection naturelle. Dans ce cas, le nombre de Rayleigh est toujours une notion utile pour prédire le mouvement du fluide à l'intérieur, à condition toutefois de le modifier pour prendre en compte la perméabilité $K_p$ ainsi que la diffusivité thermique $\kappa_p$ de ce milieu. Il vient alors l'expression du nombre de Rayleigh-Darcy noté $\mathrm{Ra}_p$, et dont la définition est donnée par Nield et Bejan \cite{nield_convection_2013} par

\begin{equation}
	\mathrm{Ra}_p = \frac{g \beta L_c K_p}{\nu \kappa_p} (T_c-T_f),
	\label{eq:NbrRayleigh_poreux}
\end{equation}
ainsi que la vitesse verticale de référence correspondante, 

\begin{equation}
	v_{\textsf{ref},p}^{// \mathbf g} = \frac{\kappa_p}{L_c} \Rayleigh_p.
	\label{eq:VitesseReference_Rayleigh_poreux}
\end{equation}


Lorsque la convection naturelle provoque un écoulement circulant à une vitesse de référence $v_{\sf ref}$, il est possible de quantifier la contribution des échanges thermiques ainsi provoqués et des pertes visqueuses en définissant le nombre de Grashof par

\begin{equation}
	\Grashof = \left( \frac{v_{\sf ref} L_c}{\nu} \right)^2,
	\label{eq:NbrGrashof}
\end{equation}
et qui est relié au nombre de Rayleigh par la formule

\begin{equation}
	\Grashof = \frac{\Rayleigh}{\Prandtl}.
	\label{eq:NbrGrashof_RayleighSurPrandtl}
\end{equation}


\paragraph{Avec acoustique}
Les termes précédents sont issus de la littérature en l'absence d'écoulement oscillant. Cette hypothèse ne peut pas être respectée dans le cas des expériences menées avec acoustique, et un autre indicateur est introduit pour quantifier les échanges de chaleur causé par un fluide en mouvement. Cet indicateur est le nombre de Péclet, noté $\Peclet$, et défini par

\begin{equation}
	\Peclet = \frac{v L_c}{\kappa}.
	\label{eq:NbrPeclet}
\end{equation}

Contrairement au nombre de Rayleigh qui sert à comparer le mouvement d'un fluide causé par un échange thermique, le nombre de Péclet quantifie les échanges de chaleur réalisés par un fluide déjà en mouvement. Cependant, il reste nécessaire de proposer une hypothèse quant à l'utilisation de ce nombre : la vitesse d'entraînement du fluide est ici la vitesse acoustique efficace $v_{\sf RMS}$, contrairement aux cas classiques de son utilisation dans la littérature où un écoulement continu est considéré. De même que le nombre de Rayleigh, le nombre de Péclet est lié aux nombres de Grashof et Prandtl suivant la relation 

\begin{equation}
	\Peclet \equiv \sqrt{\Grashof}~\Prandtl,
	\label{eq:NbrPeclet_GrashofFoisPrandtl}
\end{equation}
différente de l'équation~\eqref{eq:NbrGrashof_RayleighSurPrandtl}.\medskip

Ces échanges thermiques par convection sont à comparer aux transferts de chaleur par conduction, car il est tout à fait possible que les parois des cavités ou encore le matériau poreux lui-même offrent un chemin pour la diffusion thermique. La prépondérance de chaque effet est donnée par le calcul du nombre de Nusselt, dont la définition est

\begin{equation}
	\Nusselt = \frac{h L_c}{k},
\end{equation}
avec $h$ le coefficient d'échange convectif à déterminer.\bigskip

Les calculs des nombres précédent peuvent guider l'intuition quant aux prédominances de chaque effet prennant place dans le noyau et aux environs de celui-ci : la vitesse d'entraînement du gaz par convection, la quantité de chaleur transportée par elle, son importance par rapport à la conduction par le matériau poreux et ses parois. Des paramètres utiles pour des modèles plus avancés et présenté notamment en section~\ref{chap:ModeleTemporel} doivent également être évalués par ces équations.

\subsubsection{\'Elements finis}
\paragraph{Volume d'adaptation d'impédance}
Le modèle 2D simplifié ne prend pas en compte plusieurs paramètres : la cavité est en réalité un cône. Pour connaître l'allure des lignes de courant à l'intérieur en présence d'un flux de masse provoqué par la convection naturelle, un modèle de la cavité est réalisé dans le logiciel d'éléments finis Comsol Multiphysics grâce à une géométrie présenté sur la figure~\ref{fig:CaviteConvNat_ComsolGeometrie}. Avec ce logiciel, il est également possible de coupler une simulation acoustique en plus de la simulation de transferts thermiques, ce que le modèle 2D simplifié ne permet pas aisément.

\begin{figure}[!ht]
    \centering
    \includegraphics[width=.5\textwidth]{../fig/fig_ConvNatComsol/Geometry.png}
    \caption{Géométrie du volume d'adaptation d'impédance entre la source acoustique principale et le noyau thermoacoustique}
    \label{fig:CaviteConvNat_ComsolGeometrie}
\end{figure}

\paragraph{Régénérateur}
Le nombre de Rayleigh dans un matériau poreux dépend de sa perméabilité. Il est assez difficile de la calculer car il faut connaître la vitesse d'écoulement et la différence de pression de part et d'autre du domaine étudié. Le régénérateur est modélisé dans Comsol pour extraire ses paramètres hydrauliques et calculer les nombres adimensionnels déjà présentés.

De plus, cette simulation vise à vérifier l'hypothèse de superposition des effets thermiques. En effet, la distribution de température du côté de l'échangeur froid dans le cas de l'orientation `\texttt{H1}' laisse à penser que la température froide et constante sur la section provoqué par l'effet thermoacoustique, la conduction thermique dans le noyau qui provoque une différence de température entre le centre du noyau et sa périphérie, et la convection naturelle qui apporte un gradient de température selon la direction verticale se combinent.

\subsection{Modèle temporel} \label{chap:ModeleTemporel}
Un modèle temporel du régime transitoire de la distribution axiale de température dans le noyau thermoacoustique est créé pour approcher le réfrigérateur \textsc{Tacot} d'après le modèle 1D développé par Lotton \textit{et al.} pour un réfrigérateur à ondes stationnaires \cite{lotton_transient_2009}. Ce modèle calcule le bilan de chaleur au sein du régénérateur en faisant intervenir les flux de chaleur thermoacoustique $Q_{\sf TA}$, de conduction thermique $Q_{\sf cond}$, de frottement visqueux $Q_{\sf visq}$, et de pertes latéral au travers des parois de la cavité $Q_{\sf lat}$ dans chaque volume élémentaire $S_{\sf reg} \deriv x$ du régénérateur discrétisé. Pour compenser les écarts entre les prévisions du modèle et les mesures, un flux de chaleur $Q_{\sf vort}$ estimé empiriquement est également pris en compte dans les conditions aux frontières sur l'axe du noyau. Ce flux est supposé lié aux effets de bord du noyau tels que la vorticité, les pertes de charges ou les effets entropiques. Ce modèle 2D prend en compte les conditions au limites suivantes : 



%\begin{comment}
Le modèle a pour but de calculer les transferts thermiques dans le régénérateur pour n'importe quel champ acoustique. Aussi, l'expression des quantités oscillantes dans le noyau peut-être donnée par un produit de matrices de transfert élémentaires de l'équation~\eqref{eq:TMatrix_prod_TppTuu}. Dans le cas d'un régénérateur compact du point de vue acoustique, les coefficients de cette matrice de transfert sont donnés par

\begin{subequations}
	\begin{multicols}{2}
	\noindent
	\begin{align}
		T_{pp}^{(n)} &= 1, \label{eq:TMatrix_Tpp_regen}\\
		T_{pu}^{(n)} &= -\frac{i \omega \rho}{\Phi S (1-f_\nu)}\deriv x, \label{eq:TMatrix_Tpu_regen}
		\end{align}
		\begin{align}
		T_{up}^{(n)} &= -\frac{i \omega \Phi S}{\gamma P_0} \deriv x, \label{eq:TMatrix_Tup_regen}\\
		T_{uu}^{(n)} &= 1. \label{eq:TMatrix_Tuu_regen}
	\end{align}
	\end{multicols}
	\label{eq:TMatrix_regen}
\end{subequations}

\subsubsection{Flux thermiques}
Ensuite, les flux thermiques sont également pour la plupart différents dans le cas d'un réfrigérateur contenant un régénérateur à la place d'un \textit{stack}, et sont définis dans les équations \eqref{eq:FluxTA_Lotton_Regen} à \eqref{eq:FluxLat_Lotton} accompagnées de la figure~\ref{fig:Schema_FluxThermiquesNoyau_Gaelle} pour les illustrer au sein du régénérateur.

\begin{figure}[!ht]
    \centering
    \external{fig_Schema_FluxThermiquesNoyau_Gaelle}
%    \externalremake
    \begin{tikzpicture}%[yscale=3]
	\def\Lreg{6cm};
	\def\Ryreg{4cm};
	\def\Rxreg{1cm};%{\Ryreg/3};
	\def\CHXnorth{0,\Ryreg};
	\def\CHXsouth{0,-\Ryreg};
	\def\AHXnorth{\Lreg,\Ryreg};
	\def\AHXsouth{\Lreg,-\Ryreg};
	

	


% --------------------------------------------------- Flux thermiques
\fill[MatlabBlue,opacity=.5] (\CHXnorth) arc [start angle=90, end angle=270, x radius=\Rxreg, y radius=\Ryreg] -- cycle; % Face gauche (fond)
\draw[very thick] (\CHXnorth) arc [start angle=90, end angle=270, x radius=\Rxreg, y radius=\Ryreg]; % Face gauche (contour)

%%% Q_vort
\foreach \y in {-.5,0,.5}{
\draw[arw,->,draw=MatlabPurple,line width=1mm] (-.1*\Rxreg,{(\y+.02)*\Ryreg}) -- ++(-\Rxreg,0) arc (-90:-360:.35);
\draw[arw,->,draw=MatlabPurple,line width=1mm] (-.1*\Rxreg,{(\y-.02)*\Ryreg}) -- ++(-\Rxreg,0) arc (90:360:.35);
\draw[arw,->,draw=MatlabPurple,line width=1mm] (\Lreg+.1*\Rxreg,{(\y+.02)*\Ryreg}) -- ++(\Rxreg,0) arc (-90:180:.35);
\draw[arw,->,draw=MatlabPurple,line width=1mm] (\Lreg+.1*\Rxreg,{(\y-.02)*\Ryreg}) -- ++(\Rxreg,0) arc (90:-180:.35);}

\draw (-1.2*\Rxreg,-.5*\Ryreg) node[left]{$Q_{\sf vort}(0)$};
\draw (\Lreg+1.2*\Rxreg,.5*\Ryreg) node[right]{$Q_{\sf vort}(L_{\sf reg})$};

% --------------------------------------------------- Fond de schéma

\draw[dotted, ->, very thick] ({-3.5*\Rxreg},0) -- ({\Lreg+3.5*\Rxreg},0) node [right] {$\mathbf e_x$}; % axe x
\fill[MatlabBlue,opacity=.5] (\CHXnorth) arc [start angle=90, end angle=-90, x radius=\Rxreg, y radius=\Ryreg] -- cycle; % bord gauche (moitié droite pour que l'axe x rentre "dans le noyau")
\draw[dotted, ->, very thick] ($(\CHXsouth)+(0,-.3*\Ryreg)$) -- ($(\CHXnorth)+(0,.3*\Ryreg)$) node [above] {$\mathbf e_r$}; % axe r

\draw[dashed, very thick] (\AHXnorth) arc [start angle=90, end angle=270, x radius=\Rxreg, y radius=\Ryreg]; % paroi gauche arrière pointillée
\filldraw[fill=MatlabBlue, fill opacity=.5, draw=black,very thick] (\CHXnorth) arc [start angle=90, end angle=-90, x radius=\Rxreg, y radius=\Ryreg] -- (\AHXsouth) arc [start angle=-90, end angle=90, x radius=\Rxreg, y radius=\Ryreg] --cycle; % Paroi du cylindre


% --------------------------------------------------- Dimensions
\draw (\CHXnorth) node [above right] {\begin{tabular}{ll}\'Echangeur \\  froid\end{tabular}};
\draw (\AHXnorth) node [above left] {\begin{tabular}{rr}\'Echangeur \\ ambiant \end{tabular}  };

\draw (0,0) node[above right] {0};
\draw (\Lreg,0) node{|} node[below] {$L_{\sf reg}$};
\draw (\CHXnorth) -- ++(-.5cm,0) node[left] {$R_{\sf reg}$};

%%% Q_TA
\draw[arw,->,draw=MatlabBlue] ({.25*\Lreg},0) -- ++(.33*\Lreg,0) node[right] {$Q_{TA}$} node(MidQTA)[midway]{};
%
%\draw[arw,<-,draw=MatlabBlue] ($(\CHXsouth)+(-1mm,-1mm)$) -- ++(-120:.33*\Lreg)node[pos=.75, left]{$Q_{TA}(0)$};
%\draw[arw,->,draw=MatlabBlue] ($(\AHXsouth)+(1mm,-1mm)$) -- ++(-60:.33*\Lreg)node[pos=.25, right]{$Q_{TA}(L_{\sf reg})$};

%%% Q_cond
\draw[arw,->,draw=MatlabOrange] ({.75*\Lreg},.15*\Ryreg)  -- ++(-.33*\Lreg,0) node[left] {$Q_{\sf cond}^{(x)}$} node(MidCondx1)[midway]{};
\draw[arw,->,draw=MatlabOrange] ({.75*\Lreg},-.15*\Ryreg)  -- ++(-.33*\Lreg,0) node(MidCondx2)[midway]{} node[left] {$Q_{\sf cond}^{(x)}$};

\draw[arw,->,draw=MatlabYellow] ({.25*\Lreg},.3*\Ryreg)  -- ++(0,.33*\Lreg) node[above] {$Q_{\sf cond}^{(r)}$} node(MidCondr1)[midway]{};
\draw[arw,->,draw=MatlabYellow] ({.75*\Lreg},-.3*\Ryreg)  -- ++(0,-.33*\Lreg) node[below] {$Q_{\sf cond}^{(r)}$} node(MidCondr2)[midway]{};

%%% Q_visq
\draw[arw,<->,draw=PythonGreen] ($(MidCondr1 -| MidCondx1)+(-.33*\Lreg/2,0)$)  -- ++(.33*\Lreg,0) node[midway,above]{$Q_{\sf visq}$};
\draw[arw,<->,draw=PythonGreen] ($(MidCondr2 -| MidQTA)+(-.33*\Lreg/2,0)$)  -- ++(.33*\Lreg,0) node[midway,below]{$Q_{\sf visq}$};


%%% Q_lat
\draw[->,line width=1mm,decorate,decoration={snake,amplitude=.4mm,segment length=2mm,post length=2mm},red] (.5*\Lreg,1.1*\Ryreg) -- ++(0,1cm) node [above] {$h_r(R_{\sf reg})\theta$};
\draw[->,line width=1mm,decorate,decoration={snake,amplitude=.4mm,segment length=2mm,post length=2mm},red] (.5*\Lreg,-1.1*\Ryreg) -- ++(0,-1cm) node [below] {$h_r(R_{\sf reg})\theta$};

\draw[->,line width=1mm,decorate,decoration={snake,amplitude=.4mm,segment length=2mm,post length=2mm},red] (-2*\Rxreg,0) -- ++(-1cm,0) node [midway, above] {$h_x(0)\theta$};
\draw[->,line width=1mm,decorate,decoration={snake,amplitude=.4mm,segment length=2mm,post length=2mm},red] (\Lreg+2*\Rxreg,0) -- ++(1cm,0) node [midway, above] {$h_x(L_{\sf reg})\theta$};

%%% Q_HX
%\draw[arw,<-,draw=gray] ($(\CHXsouth)+(1mm,-1mm)$) -- ++(-60:.33*\Lreg)node[pos=.75, right]{$Q_f$};
%\draw[arw,->,draw=gray!50!black] ($(\AHXsouth)+(-1mm,-1mm)$) -- ++(-120:.33*\Lreg)node[pos=.25, left]{$Q_a$};
\draw[arw,<-,draw=gray] ($(\CHXsouth)+(0,-1mm)$) -- ++(-90:.33*\Lreg)node[pos=.75, right]{$Q_f$};
\draw[arw,->,draw=gray!50!black] ($(\AHXsouth)+(0,-1mm)$) -- ++(-90:.33*\Lreg)node[pos=.25, left]{$Q_a$};

\end{tikzpicture}
    \caption{Représentation schématique des flux thermiques considérés dans le modèle transitoire du régénérateur.}
    \label{fig:Schema_FluxThermiquesNoyau_Gaelle}
\end{figure}

\begin{enumerate}[label=\textbf{(\roman*)}]
\item Tout d'abord, le flux thermoacoustique 1D est calculé avec l'équation~\eqref{eq:FluxTA_swift} développée par Swift \cite{swift_thermoacoustics_2017}

\begin{multline}
	Q_{TA} = \underbrace{ -\frac12~\RE\left[ \frac{ f_\kappa-f_\nu^* }{ (1+\mathrm{Pr})(1-f_\nu^*) } pu^* \right] }_{q_{TA}(x)} \\ 
	+ \overbrace{ \frac12 \frac{\rho_0 C_p}{\Phi S}~\IM\left[ \frac{f_\kappa + \mathrm{Pr} f_\nu^*}{(1-\mathrm{Pr}^2)|1-f_\nu|^2} \right] |u|^2 }^{k_{TA}(x)}\partial_x \theta(r,x;t),
\label{eq:FluxTA_Lotton_Regen}
\end{multline}
où $\theta(r,x;t) = T_0(r,x;t)-T_\infty$ est l'écart de la température locale à la température initiale. Pour la suite du document, cet écart de température sera simplement écrit $\theta$, sans la dépendance spatio-temporelle. Il est important de remarquer la partie réelle de ce flux thermique, notée $q_{TA}$, qui ne dépend que du champ acoustique, et la partie imaginaire $k_{TA}$ qui dépend du gradient de température le long du stack et qui peut être vu comme un terme de conduction induite par l'effet thermoacoustique.

\item Ensuite, la conduction dans le régénérateur est prise en compte dans les directions $\textbf e_x$ et $\textbf{e}_r$ du régénérateur avec l'équation exprimée en coordonnées cylindriques par

%\begin{equation}
%	Q_{\sf cond} = -\begin{pmatrix}
%	k_x\\
%	k_r\\
%	k_\alpha
%	\end{pmatrix} \cdot \nabla \theta,
%	\label{eq:FluxCond_Lotton}
%\end{equation}

\begin{equation}
	\left\{ \begin{aligned}
		Q_{\sf cond}^{(x)} &= -k_x \partial_x T\\
		Q_{\sf cond}^{(r)} &= -k_r \partial_r T \echaf{\text{ corriger expression avec gradient radial}}
	\end{aligned}\right.
	\label{eq:FluxCond_Lotton}
\end{equation}
avec les indices $x$ et $r$ dénotant respectivement les directions axiale et radiale où circulent les flux de conduction. La porosité influe sur la conductivité thermique apparente et $k_i~=~\Phi_i k_{g} + (1-\Phi_i)k_{s}$ ($i=x,r$). \echaf{En première aproximation, il est possible de considérer les valeurs de conductivité comme égales les unes aux autres. Cependant, c'est inexact car le long de l'axe $\mathbf e_x$ le milieu est constitué de l'empillement de tissus métalliques tandis que dans la direction $\mathbf e_r$ le flux de chaleur parcourt les fils qui forment le tissus en lui-même.}

%\begin{equation}
%Q_{\sf cond}^{(x)} = -k_x\partial_x\theta~\mathrm{\mathbf e_x},
%\label{eq:FluxCondx_Lotton}
%\end{equation}
%avec... 
%
%\begin{equation}
%Q_{\sf cond}^{(r)} = -k_r\partial_r\theta~\mathrm{\mathbf e_r},
%\label{eq:FluxCondr_Lotton}
%\end{equation}
%
%\begin{equation}
%Q_{\sf cond}^{(\alpha)} = -k_\alpha\partial_\alpha\theta~\mathrm{\mathbf e_\alpha},
%\label{eq:FluxCondalpha_Lotton}
%\end{equation}
%\textcolor{red}{Rassembler $Q_{cond}$ en 1 seul terme} avec ...

\item Les pertes visqueuses sont estimées avec l'équation

\begin{equation}
Q_{\sf visq} = \frac{1}{R_{\sf reg}^2} \int_{0}^{R_{\sf reg}} \frac{1}{\tau_0} \int_0^{\tau_0} \mu (\partial_r u)^2\ r \deriv r \deriv t.
\label{eq:FluxVisq_Lotton}
\end{equation}

En utilisant l'identité

\begin{equation*}
\frac{1}{T}\int_{0}^{\tau_0}\mu(\partial_r u)^2\deriv t = \frac12|\partial_r u|^2,
\end{equation*}
l'équation \eqref{eq:FluxVisq_Lotton} est réécrite comme

\begin{equation}
	Q_{\sf visq} = \frac{1}{2R_{\sf reg}^2} \int_{0}^{R_{\sf reg}} \mu |\partial_r u|^2\ r \deriv r
	\label{eq:FluxVisq_Lotton_IntegralTSimplifie}
\end{equation}

\echaf{poursuivre après simu}
dont l'intégrale sur la section de rayon $R_{\sf reg}$ est résolue par une méthode numérique.

\item Les effets de bords tels que les pertes de charges ou la vorticité sont pris en compte aux frontières du domaine grâce à des flux de chaleur $Q_{\sf vort}$ estimés empiriquement à $x=0$ et $x=L_{\sf reg}$.

\item Enfin, les pertes latérales sur la circonférence et aux extrémités du régénérateur sont prises en compte par

\begin{equation}
	\left\{ \begin{aligned}
		Q_{\sf lat}^{(x)} &= h_x \theta,\\
		Q_{\sf lat}^{(x)} &= h_x \theta,
	\end{aligned}\right.
\label{eq:FluxLat_Lotton}
\end{equation}
où $h_x$ et $h_r$ sont respectivement les coefficients d'échanges avec les extrémités à $x=0$ et $x=L_{\sf reg}$, et avec la paroi à $r = R_{\sf reg}$. Ils sont déterminés de façon empirique dans l'article.

%Alternativement, il est possible de prendre en compte les échanges de chaleur au bord et dans l'axe du noyau par un terme qui représente le flux de chaleur induit par les oscillations acoustiques noté $\psi_{ac}$ dans la thèse de Guédra \cite{guedra_etudes_2012} et défini par
%
%\begin{equation}
%	\psi_{ac}=\frac12 \rho_0 T_0 \RE[s v^*],
%	\label{eq:FluxChaleurAcou_Guedra}
%\end{equation}
%dont la moyenne sur la section s'écrit
%
%\begin{align}
%	\langle\psi_{ac}\rangle &= \RE[g]\mathcal I \IM[g]\mathcal J - k_{ac}\deriv_x T_0,	\label{eq:FluxChaleurAcou_Guedra_Moy}\\
%										&= \langle\psi_{ac,0}\rangle
%\end{align}

\end{enumerate} 

\subsubsection{Conditions aux limites}
\begin{enumerate}[label=\textbf{(\roman*)}]
\item \echaf{Condition de température sur le canister}


\begin{figure}[!ht]
    \centering
    \external{fig_TemperatureCanisterConditionLimite}
    %\externalremake
    \begin{tikzpicture}
	\def\lREG{3.9cm};
	\def\eCAN{5mm};
	\def\eWALL{13pt};
	\def\eFB{6mm};
	
	\fill[gray] (0,0) node(O){} rectangle (\lREG,-\eCAN);
	\fill[top color=gray,bottom color=white](0,-\eCAN) node(eCAN){} rectangle (\lREG,-2*\eCAN);
	
	\fill[right color=black!25,left color=white] ($(O)+(0,\eWALL)$) node(eWALL){} rectangle (-\lREG/4,0);
	\fill[black!25] (O) rectangle (\lREG,\eWALL) node[midway]{$k_{\sf acier}$};
	\fill[right color=white,left color=black!25] ($(O)+(\lREG,0)$) rectangle ++(\lREG/4,\eWALL);
	
	\fill[blue!25!white] ($(O)+(0,\eWALL+\eFB)$) node(eFB){} rectangle ++(\lREG,-\eFB);
	\fill[right color=blue!25!white,left color=white] (eFB) rectangle ++(-\lREG/4,-\eFB);
	\fill[right color=white,left color=blue!25!white] ($(eFB)+(\lREG,0)$) rectangle ++(\lREG/4,-\eFB);
	\fill[top color=white,bottom color=blue!25!white] (eFB) rectangle ++(\lREG,.5*\eFB);
	\shade[upper left=white,upper right=white,
         lower left=white,lower right=blue!25!white] (eFB) rectangle ++(-\lREG/4,.5*\eFB);
	\shade[upper left=white,upper right=white,
         lower left=blue!25!white,lower right=white] ($(eFB)+(\lREG,0)$) rectangle ++(\lREG/4,.5*\eFB);
	\draw[->,line width=.5mm,decorate,decoration={snake,amplitude=.4mm,segment length=2mm,post length=2mm},blue!75!white] ($(eWALL)+(\lREG/2,0)$) -- ++(0,\eFB) node[midway,right,blue!75!white]{$h_{\sf conv}$};
	
	
	
	\draw[->] ($(eFB)+(0,\eFB)$) -- ($(eCAN)+(0,-2.1*\eCAN)$) node[left]{$-\mathbf{e}_r$};
	\draw[->] (-\lREG/3,0) node(lab){} -- (4*\lREG/3,0) node[above]{$\mathbf{e}_x$};
	
	\draw ($(lab) + (0,\eWALL+1.5*\eFB/2)$) node[left]{Canal de rétroaction};
	\draw ($(lab) + (0,\eWALL/2)$) node[left]{Tube contenant le noyau};
	\draw (eWALL) node[left]{$e_{\sf wall}$};
	\draw (O) node[left]{0};
	\draw[line width=1mm,dashed] (eCAN.base) node[left]{$e_{\sf can}$} -- ++(\lREG,0);
\end{tikzpicture}
    \caption{Géométrie équivalente pour établir la condition de température à la frontière $r=R_{\sf reg}$.}
    \label{fig:TemperatureCanisterConditionLimite}
\end{figure}

\item \echaf{Flux de chaleur apportés ou retirés par les échangeurs}

$Q=\epsilon Q_i$ avec 
\begin{equation}
	\epsilon = \frac{k_s k_{sfrac} (1-\Phi)S_{\sf reg}}{k_s k_{sfrac} (1-\Phi)S_{\sf reg} + k_{\sf can} S_{\sf can}}
\end{equation}
et $k_{sfrac}$ le facteur de dégradation de conductivité \cite{lewis_measurement_1998}

\end{enumerate}

\subsubsection{Bilan thermique}

Ces flux de chaleurs sont reportés dans l'équation modifiée du bilan de chaleur qui s'écrit

%\begin{equation}
%[\Phi\rho_0 C_p + (1-\Phi)\rho_s C_s]\partial_t \theta = - \nabla \cdot (Q_{TA} + Q_{\sf cond}) + Q_{\sf visq},
%\label{eq:BilanChaleur_LottonPoignand}
%\end{equation}
%ce qui donne après développement de l'opérateur divergence en coordonnées cylindrique,

\begin{equation}
[\Phi\rho_0 C_p + (1-\Phi)\rho_s C_s]\ \partial_t \theta = - \partial_x Q_{TA} - \partial_x Q_{\sf cond}^{(x)} - \frac{1}{r}\partial_r[r Q_{\sf cond}^{(r)}] + Q_{\sf visq}.
\label{eq:BilanChaleur_LottonPoignand_ApDvptCylin}
\end{equation}

Le premier membre désigne la variation d'énergie interne suite à la variation de température, et le second membre, l'expression des flux thermiques dans chaque volume élémentaire. D'une part, les flux thermoacoustiques et de conduction sont \og de passage \fg{} au travers de ce volume, et c'est seulement dans le cas d'une divergence non nulle que la température y évolue. D'autre part, les effets visqueux se produisent dans chaque volume, il sont donc source de chaleur et doivent être pris en compte tels quels. Ceci dit, le régénérateur est supposé axisymétrique, ce qui permet de considérer les composantes azimuthales $\nabla_\alpha \cdot Q_{TA}$ et $\nabla_\alpha \cdot Q_{\sf cond}$ comme nulles.

Le problème décrit par l'équation~\ref{eq:BilanChaleur_LottonPoignand_ApDvptCylin} est conditionné aux frontières en $x=0$, $x=L_{\sf reg}$ et $r=R_{\sf reg}$ pour respecter la conservation d'énergie. Ces conditions se traduisent par le système d'équations

\begin{subequations}
	\begin{align}
		Q_{\sf cond}^{(x)} + Q_f + Q_{\sf vort}(0) &= Q_{TA}(0) + h_x(0)\theta,\quad &x=0,\ r \in [0,R_{\sf reg}], t>0 \label{eq:CondFront_x_0Froid}\\%
		Q_{TA}(L_{reg}) + Q_{\sf vort}(L_{reg})  &= Q_{\sf cond}^{(x)} + Q_a + h_x(L_{\sf reg})\theta, \quad &x=L_{\sf reg},\ r \in [0,R_{\sf reg}], t>0 \label{eq:CondFront_x_LregAmb}\\%
		Q_{\sf cond}^{(r)} &= h_r(R_{\sf reg})\theta, \quad &r=R_{\sf reg},\ x \in [0,L_{\sf reg}], t>0 \label{eq:CondFront_r_ra}
	\end{align}%
	\label{eq:CondFront}%
\end{subequations}
où $Q_f$ et $Q_a$ sont les flux de chaleur retirés à l'échangeur froid d'une part et apporté à l'échangeur ambiant d'autre part. Ce système signifie que pour chaque frontière, la somme des flux entrant dans l'interface et des sources thermiques est égale aux flux qui en sortent. Il reste à introduire la distribution de température à $t=0$ notée $\theta_0(x,r)$ comme condition initiale, qui se présente comme

\begin{equation}
	\theta(x,r,t=0) = \theta_0(x,r).
	\label{eq:CondInit}
\end{equation}

Le report des flux de chaleurs des équations~\eqref{eq:FluxTA_Lotton_Regen} à \eqref{eq:FluxLat_Lotton} dans l'équation du problème~\eqref{eq:BilanChaleur_LottonPoignand_ApDvptCylin}, dans les conditions aux frontières~\eqref{eq:CondFront} et dans la condition initiale~\eqref{eq:CondInit} permet de poser le problème tel que

\begin{equation}
	toto
	\label{eq:ProbTransitoire_ApDvpt}
\end{equation}

La résolution de l'équation~\eqref{eq:BilanChaleur_LottonPoignand_ApDvptCylin} qui comporte des termes sources requiert l'usage de la méthode des transformations intégrales après réécriture des équations du système sous la forme d'un problème homogène équivalent \cite{ozisik_heat_1993}. Les équations s'écrivent alors

\begin{subequations}
	\begin{align}
		\echaf{Systeme\ ici},\label{eq:1_LottonPoignand}\\
		\echaf{Systeme\ ici},\label{eq:2_LottonPoignand}
	\end{align}
	\label{eq:SystemeEq_LottonPoignand}
\end{subequations}
avec les changements de variables suivants

\begin{itemize} \color{red}
	\item 1
	\item 2.
\end{itemize}

%\end{comment}







