\chapter{Dispositif expérimental}\label{chap:DispositifExpe}
\mylocaltoc

%\vfill
%\begin{figure}[!ht]
%    \centering
%    \external{fig_TOC_DispositifExpe}
%%    \externalremake
%    \begin{tikzpicture}[mindmap,concept color=MatlabBlue, text=black,
%%	every concept/.append style={ellipse},
	level 1 concept/.append style={font=\small}]
	\node [concept] {Plan du chapitre} %[clockwise from=-60]
    	child[grow=-45,concept color=MatlabOrange] {node[concept] {Présentation du dispositif} [clockwise from=0]
    		child[concept color=MatlabOrange!75] {node[concept](Res) {Résultats déjà obtenus}}
    		child[concept color=MatlabOrange!75] {node[concept] {Géométrie du \textsc{Tacot}} [clockwise from=-45]
    			child[concept color=MatlabOrange!50] {node[concept] {Cavité}}
    			child[concept color=MatlabOrange!50] {node[concept] {Noyau}}}
    		child[concept color=MatlabOrange!75] {node[concept] {Instrumen\-tation}
    			[clockwise from=-75] 
    			child[concept color=MatlabOrange!50] {node[concept] {Exci\-tation}}
    			child[concept color=MatlabOrange!50] {node[concept] {Acqui\-sition}}}}
    	child[grow=-135,concept color=MatlabYellow] {node[concept] {Protocole expérimental}
    		[clockwise from=-60]
    		child[concept color=MatlabYellow!75] {node[concept] {Orien\-tations}
    			[clockwise from=-45]
    			child[concept color=MatlabYellow!50] {node[concept] (H1) {`\texttt{H1}'}}
    			child[concept color=MatlabYellow!50] {node[concept] {`\texttt{H2}'}}
    			child[concept color=MatlabYellow!50] {node[concept] {`\texttt{V1}'}}
    			child[concept color=MatlabYellow!50] {node[concept] {`\texttt{V2}'}}
    			} 
    		child[concept color=MatlabYellow!75] {node[concept] {Acqui\-sitions}
    			[clockwise from=-105]
    			child[concept color=MatlabYellow!50] {node[concept] {Avec acoustique}}
    			child[concept color=MatlabYellow!50] {node[concept] {Sans acoustique}}}};
\begin{pgfonlayer}{background}
    \draw [concept connection]  (H1) edge (Res);
  \end{pgfonlayer}
\end{tikzpicture}		
%%    \caption{•}
%%    \label{fig:•}
%\end{figure}
%\vfill

\newpage

%\vfill
%{
%\tikzsetfigurename{fig_TOC_DispositifExpe}
%\begin{tikzpicture}[mindmap,concept color=MatlabBlue, text=black,
%%	every concept/.append style={ellipse},
	level 1 concept/.append style={font=\small}]
	\node [concept] {Plan du chapitre} %[clockwise from=-60]
    	child[grow=-45,concept color=MatlabOrange] {node[concept] {Présentation du dispositif} [clockwise from=0]
    		child[concept color=MatlabOrange!75] {node[concept](Res) {Résultats déjà obtenus}}
    		child[concept color=MatlabOrange!75] {node[concept] {Géométrie du \textsc{Tacot}} [clockwise from=-45]
    			child[concept color=MatlabOrange!50] {node[concept] {Cavité}}
    			child[concept color=MatlabOrange!50] {node[concept] {Noyau}}}
    		child[concept color=MatlabOrange!75] {node[concept] {Instrumen\-tation}
    			[clockwise from=-75] 
    			child[concept color=MatlabOrange!50] {node[concept] {Exci\-tation}}
    			child[concept color=MatlabOrange!50] {node[concept] {Acqui\-sition}}}}
    	child[grow=-135,concept color=MatlabYellow] {node[concept] {Protocole expérimental}
    		[clockwise from=-60]
    		child[concept color=MatlabYellow!75] {node[concept] {Orien\-tations}
    			[clockwise from=-45]
    			child[concept color=MatlabYellow!50] {node[concept] (H1) {`\texttt{H1}'}}
    			child[concept color=MatlabYellow!50] {node[concept] {`\texttt{H2}'}}
    			child[concept color=MatlabYellow!50] {node[concept] {`\texttt{V1}'}}
    			child[concept color=MatlabYellow!50] {node[concept] {`\texttt{V2}'}}
    			} 
    		child[concept color=MatlabYellow!75] {node[concept] {Acqui\-sitions}
    			[clockwise from=-105]
    			child[concept color=MatlabYellow!50] {node[concept] {Avec acoustique}}
    			child[concept color=MatlabYellow!50] {node[concept] {Sans acoustique}}}};
\begin{pgfonlayer}{background}
    \draw [concept connection]  (H1) edge (Res);
  \end{pgfonlayer}
\end{tikzpicture}		
%}
%\vfill
%\newpage

\section{Introduction}\label{chap:IntroProtocExp}
Après avoir brièvement rappelé les bases de fonctionnement des machines thermoacoustiques, il est temps de décrire le réfrigérateur support de cette thèse. Le choix de la géométrie, les paramètres hydrauliques du régénérateur utilisé, et enfin la chaîne d'excitation et d'acquisition sont présentés dans la section~\ref{chap:PresentationTacot} - \nameref{chap:PresentationTacot}. La deuxième partie présente les conditions expérimentales choisies ainsi que le protocole suivi pour chaque mesure dans la section~\ref{chap:ProtocolExpe} - \nameref{chap:ProtocolExpe}.
%Les simulations théoriques réalisées sont ensuite expliquées en troisième partie, dans la section~\ref{chap:SimusRealisees} (\nameref{chap:SimusRealisees}).

%Ces parties visent à créer une vue d'ensemble de ce qui est fait durant cette thèse et le présenter de manière globale pour pouvoir s'y référer dans les différents chapitres suivants.

\section{Présentation du dispositif expérimental actuel}\label{chap:PresentationTacot}

\subsection{Résultats déjà obtenus}\label{chap:PresTacot_ResultatsATE}
\echaf{Quelques résultats. À mettre dans l'intro I ?} Au démarrage de cette thèse, le réfrigérateur existe déjà et sa caractérisation est publiée \cite{ramadan_design_2021}, et fait également l'objet d'études expérimentales~\cite{ramadan_experimental_2018, ramadan_experimental_2021} et  numériques~\cite{hireche_numerical_2019, hireche_experimental_2020, baltean_gravity_2025}. 

Les expériences réalisées avant le début de cette thèse sont faite en utilisant un noyau dont la porosité est $\Phi=\qty{75}{\percent}$. Il est immergé dans un mélange de gaz pressurisé à \qty{40}{\bar} et composé de \qty{70}{\percent} d'hélium et de \qty{30}{\percent} d'argon, car dans ces proportions le nombre de Prandtl du mélange est minimum \cite{belcher_working_1999}. Le modèle linéaire 1D réalisé avec le logiciel \echaf{dédié} \textsc{DeltaEC} prédit les meilleurs performances à la fréquence de résonance du système $f=\qty{47}{\hertz}$, pour laquelle la longueur d'onde vaut $\lambda=\qty{11.7}{\meter}$, validant une des hypothèse de compacité acoustique du modèle de Swift. C'est par ailleurs le seul point de fonctionnement où l'impédance électrique est supérieure à la limite basse admise par l'amplificateur, soit \qty{2}{\ohm}. Le déphasage inter-source est fixé à $\varphi_{2-1}=\ang{-60}$ pour toutes les expériences, car il s'agit de la valeur optimale d'après les simulations et ce qui est confirmé par des expériences préliminaires.

Dans cet article, il est montré qu'à ces fréquence et déphasage inter-sources, et à une amplitude $DR=\nicefrac{p_1}{p_0}=\qty{3.6}{\percent}$, une puissance d'alimentation des sources $\dot W_e=\qty{193}{\watt}$ et consommée, une quantité de chaleur de $\dot Q_f=\qty{290}{\watt}$ est extraite à la source froide et la température $T_a=\qty{15}{\degreeCelsius}$ est atteinte. Le coefficient de performance à ce point de fonctionnement est $\COP=\nicefrac{\dot Q_f}{\dot W_e}=\num{1.5}$, soit \qty{15}{\percent} du coefficient de performance de Carnot.\smallskip

Cependant, les résultats obtenus avant le démarrage de cette thèse montrent des écarts avec les prédictions de la théorie linéaire. Par exemple\echaf{ajouter figures des écarts mesures-modèle DeltaEC}

\begin{figure}[!ht]
    \centering
	\begin{subfigure}{.47\textwidth}
		\centering
		\external{fig_ATE_ProfilsAX_5mm}
%		\externalremake
		\tikz{\draw(0,0) node{\includegraphics[width=.95\textwidth]{../fig/fig_Litterature/ATE_ProfilsAX_5mm.png}};}
		\caption{}
		\label{fig:ATE_ProfilsAX_5mm}
	\end{subfigure}		
	\begin{subfigure}{.47\textwidth}
		\centering
		\external{fig_ATE_ProfilsCHX_5mm}
%		\externalremake
		\tikz{\draw(0,0) node{\includegraphics[width=.95\textwidth]{../fig/fig_Litterature/ATE_ProfilsCHX_5mm.png}};}
		\caption{}
		\label{fig:ATE_ProfilsCHX_5mm}
	\end{subfigure}	    
    \caption{Mesures et simulation linéaire de profils de température dans le régénérateur du \textsc{Tacot} pour différentes amplitudes acoustiques \cite{ramadan_design_2021}. \subref{fig:ATE_ProfilsAX_5mm} profils axiaux, et \subref{fig:ATE_ProfilsCHX_5mm}, profils transverses du côté de l'échangeur froid.}
    \label{fig:ATE_Profils_5mm}
\end{figure}

\subsection{Géométrie du réfrigérateur \textsc{Tacot}}
\subsubsection{Cavité thermoacoustique}
La pompe à chaleur a été dimensionnée et fabriquée dans le cadre du projet ANR \textsc{Tacot} (ThermoAcoustic Cooler for Onroad Transportation), qui porte sur l'application d'une pompe à chaleur thermoacoustique pour la climatisation automobile \cite{ANR_thermo-acoustic_2019}. Ce projet apporte beaucoup de contraintes, dont l'une des principales est la compacité. Contrairement aux autres systèmes thermoacoustiques existant et bien plus volumineux (tels que le liquéfacteur de gaz naturel développé par Swift et Wollan au Los Alamos National Laboratory \cite{swift_thermoacoustics_2002, wollan_development_2002}, ou le réfrigérateur cryogénique thermoacoustique spatial (STAR) \cite{adeff_measurement_1991, garrett_thermoacoustic_1993}), les dimensions doivent être réduites tout en conservant un pompage de chaleur efficace. Pour cela, une géométrie coaxiale pour la cavité thermoacoustique est préférée à celle toroïdale usuellement utilisée en suivant les travaux de Poignand \textit{et al.} \cite{poignand_thermoacoustic_2011, poignand_analysis_2013}. L'ajout d'une source acoustique secondaire dans la cavité thermoacoustique permet également de gagner en compacité, en remplaçant un résonateur plus long par la masse de son équipage mobile et la souplesse de sa suspension, tel que réalisé dans les travaux de Poese \textit{et al.} \cite{poese_thermoacoustic_2004}. En plus de permettre une diminution du volume de la machine, utiliser une source secondaire offre plus de flexibilité qu'un résonateur sur la relation entre pression acoustique et vitesse particulaire, et facilite en particulier le ciblage du déphasage optimal entre pression et vitesse acoustiques au sein du noyau thermoacoustique. Un schéma général présente la géométrie de la pompe à chaleur sur la figure~\ref{fig:SchemaGeneralTACOT}, adapté de Ramadan \textit{et al.} \cite{ramadan_design_2021}. 

\begin{figure}[!ht]
    \centering
    \external{fig_SchemaGeneralTACOT}
%    \externalremake
    \begin{tikzpicture}[scale=.5]
	\node[anchor=south west, inner sep=0] (image) at (0,0) {\includegraphics[angle=0,origin=c,width=.6\textwidth]{../fig/fig_TACOTSchematics/TACOT.png}};
	
	\begin{scope}[x={(image.south east)},y={(image.north west)}]
	
%		\filldraw (0,0) circle (2pt); 
%		\filldraw[green] (1,0) circle (1pt);
%		\filldraw[red] (1,1) circle (1pt);		
%		\filldraw[blue] (0,1) circle (1pt);
%		\draw[help lines,xstep=.1,ystep=.1] (0,0) grid (1,1);
%		\fill[orange, rounded corners, opacity=1,draw=orange] (.46,.65) -- ++(132:.09) -- ++(0,-.44) -- ++(48:.09) -- cycle;
		\draw[MatlabYellow,rounded corners,very thick,preaction={fill=MatlabYellow!20,opacity=.5}] (.46,.65) -- ++(132:.09) -- ++(-.03,0) -- ++(0,-.44) --++(.03,0) -- ++(48:.09) -- cycle; %node[left,pos=.5,label={[rotate=90]center:Cavité}]{};
		\draw[MatlabYellow] (.415,0.5) node [label={[rotate=90]center:\textbf{Cône}}]{};
		
%		\draw[blue] (.5,.5) node [anchor=center, preaction={fill=black!20,opacity=.7}] {RIX};
%		\draw[red] (.33,.5) node [anchor=center, preaction={fill=black!20,opacity=.7}] {TA core};	
	
		\draw[MatlabPurple,rounded corners,very thick,preaction={fill=MatlabPurple!20,opacity=.5}] (.365,.7) rectangle (.245,.3) node[pos=.5,label={[rotate=90]center:\textbf{Noyau}}]{};
		
		\node (AHX) at (.27,.65) {};
		\node (Reg) at (.32,.65) {};
		\node (CHX) at (.36,.65) {};
		\node (RIX) at (.77,.4) {};
		\node (HP) at (.19,.4) {};
		
		\draw[<-,very thick,MatlabOrange] (AHX.center) to[out=90,in=0] ($(AHX)+(-.15,.4)$) node[left]{\'Echangeur de chaleur ambiant};		
		\draw[<-,very thick] (Reg.center) to ($(Reg)+(0,.4)$) node[above]{Régénérateur};
		\draw[<-,very thick,MatlabBlue] (CHX.center) to[out=90,in=180] ($(CHX)+(.15,.4)$) node[right]{\'Echangeur de chaleur froid};
		
		\draw[->,very thick,green!50!black] ($(RIX)+(0,-.4)$) -- (RIX.center) node[pos=0,anchor=north]{Source acoustique principale};
		\draw[->,very thick,green!50!black] ($(HP)+(0,-.4)$) -- (HP.center) node[pos=0,anchor=north]{Source acoustique secondaire};
		
%		\draw [white] (.455,.5) node{+};
%		\draw [white] (.41,.65) node{+};
%		\draw [white] (.41,.5) node{+};
%		\draw [white] (.41,.35) node{+};

	\end{scope}

	
\end{tikzpicture}
    \caption{Schéma général du réfrigérateur \textsc{Tacot}. Le noyau thermoacoustique et le cône d'adaptation d'impédance sont mis en évidence}
    \label{fig:SchemaGeneralTACOT}
\end{figure}

\subsubsection{Noyau thermoacoustique}
Tout comme la machine qui le contient, le noyau adopte une géométrie cylindrique. Il est composé d'un régénérateur représenté sur la figure~\ref{fig:TacotPhotosNoyau_Regen} encadré par deux échangeurs de chaleur représentés sur les figures~\ref{fig:TacotPhotosNoyau_AHX} et \subref{fig:TacotPhotosNoyau_CHX}. Le premier est l'échangeur ambiant et a pour rôle d'extraire la chaleur qui s'accumule de ce côté du noyau, afin d'éviter l'échauffement global de la machine. Le second est l'échangeur froid, et sa fonction et de simuler une charge thermique à refroidir. 

\begin{figure}[!ht]
    \centering
	\begin{subfigure}{.32\textwidth}
		\centering
		\external{fig_TacotPhotosNoyau_AHX}
%		\externalremake
		\tikz{\draw(0,0) node{\includegraphics[width=.95\textwidth]{../fig/fig_TacotPhotos/AHX.JPG}};}
		\caption{}
		\label{fig:TacotPhotosNoyau_AHX}
	\end{subfigure}		
	\begin{subfigure}{.32\textwidth}
		\centering
		\external{fig_TacotPhotosNoyau_Regenerateur}
%		\externalremake
		\tikz{\draw(0,0) node{\includegraphics[width=.95\textwidth]{../fig/fig_TacotPhotos/Regenerateur.JPG}};}
		\caption{}
		\label{fig:TacotPhotosNoyau_Regen}
	\end{subfigure}	
	\begin{subfigure}{.32\textwidth}
		\centering
		\external{fig_TacotPhotosNoyau_CHX}
%		\externalremake
		\tikz{\draw(0,0) node{\includegraphics[width=.95\textwidth]{../fig/fig_TacotPhotos/CHX.JPG}};}
		\caption{}
		\label{fig:TacotPhotosNoyau_CHX}
	\end{subfigure}	    
    \caption{Composition du noyau thermoacoustique : \subref{fig:TacotPhotosNoyau_AHX} \'Echangeur ambiant, \subref{fig:TacotPhotosNoyau_Regen} régénérateur, et \subref{fig:TacotPhotosNoyau_CHX} échangeur froid.}
    \label{fig:TacotPhotosNoyau}
\end{figure}

Ces trois éléments sont ensuite insérés dans un tube cylindrique qui se fixe sur le bâti de la machine pour maintenir l'espace nécessaire à la boucle de rétroaction, ce que montre la figure~\ref{fig:TacotPhotosNoyau_Assemble}.

\begin{figure}[!ht]
    \centering
    \external{fig_TacotPhotosNoyau_Assemble}
%	\externalremake
    \tikz{\draw(0,0) node{\includegraphics[width=.5\textwidth]{../fig/fig_TacotPhotos/NoyauMonte.JPG}};}
    \caption{Noyau assemblé}
    \label{fig:TacotPhotosNoyau_Assemble}
\end{figure}


Les axes $\mathbf e_x$ et $\mathbf e_r$ sont alors respectivement associés aux directions axiale et radiale du noyau, \echaf{J'arrive pas à relire} avec pour sens positif choisi le sens de l'échangeur froid vers l'échangeur ambiant pour le premier, et du centre du noyau vers l'extérieur pour le second. Un repère est également défini pour évaluer l'orientation du réfrigérateur dans la suite de ce manuscrit.\echaf{figure + définition}\medskip

\paragraph{Régénérateur} Le régénérateur utilisé dans le \textsc{Tacot} est composé de \echaf{combien} disques de tissus métalliques (Gantois, modèle : 102045) empilés dans une enceinte cylindrique de diamètre intérieur $D_{\sf reg}=\qty{148}{\mm}$ et de longueur $L_{\sf reg}=\qty{39}{\mm}$ pour atteindre une porosité $\Phi=\qty{68}{\percent}$. Cette porosité est définie par la relation

\begin{align}
	\Phi &= \frac{V_{\sf gaz}}{V_{\sf tot}}, \label{eq:Porosite_Volume}%\\
%		 &= \frac{V_{\sf tot}-V_{\sf metal}}{V_{\sf tot}} \nonumber\\
%		 &= 1 - \frac{m_{\sf metal}}{m_{\sf tot}} \label{eq:Porosite_Masse}
\end{align}
où $V_{\sf gaz}$ représente le volume occupé par le gaz dans le régénérateur, et $V_{\sf tot}$ le volume total du régénérateur. Ce régénérateur est différent de celui utilisé dans l'article de Ramadan \textit{et al.}~\cite{ramadan_design_2021} pour comparer les performances et le comportement du réfrigérateur si son noyau est changé \echaf{phrase à retravailler}. La masse de tissus $m_{\sf tissus}$ à utiliser pour atteindre la porosité souhaitée est déduite de cette relation et s'écrit 

\begin{align}
%		 &= \frac{V_{\sf tot}-V_{\sf tissus}}{V_{\sf tot}} \nonumber\\
%	\Phi &= 1 - \frac{m_{\sf tissus}}{m_{\sf tot}}. \label{eq:Porosite_Masse} \\
	m_{\sf tissus} &= (1-\Phi) m_{\sf tot},
\end{align}
où $m_{\sf tot}$ représente la masse d'un cylindre de mêmes dimensions que le régénérateur, intégralement constitué du même acier inoxydable que les tissus, soit de l'acier inoxydable 316L.

Le milieu ainsi constitué est poreux et tortueux car l'orientation des disques de tissus est aléatoire, et le rayon hydraulique est défini par  %\echaf{ajouter source pour justifier que les matériaux type "mousse" se comporte comme du cylindrique : UPDATE dans Swift TA unifying... chapitre 7 (tortuous porous media)}

\begin{equation}
	r_h = d_w\frac{\Phi}{4(1-\Phi)},
	\label{eq:DefRayonHydrauGantois}
\end{equation}
avec $d_w$ le diamètre du fil \cite{swift_thermoacoustics_2017}. Ce milieu poreux dispose d'une certaine capacité à laisser passer un écoulement. C'est la perméabilité notée $K_p$, qui est généralement définie par 

\begin{equation}
	K_p = \echaf{v_{\sf ref}\frac{\nu \Delta x}{\Delta P}},
	\label{eq:DefPermeabilite_Wikipedia}
\end{equation}
où $v_{\sf ref}$ est la vitesse d'écoulement d'un fluide de viscosité cinématique $\nu$ causé par un gradient de pression $\nicefrac{\Delta P}{\Delta x}$ de part et d'autre du milieu poreux \cite{nield_convection_2013}. Cependant, il existe dans la littérature des formulations de la perméabilité ne prenant en compte que la géométrie interne du milieu poreux~\cite{dullien_porous_1992}. La formulation retenue s'écrit

\begin{equation}
	K_p = \echaf{\frac{4 r_h^2 \Phi}{8}},
	\label{eq:DefPermeabilite_LISN}
\end{equation}
et donne des résultats satisfaisants pour le developpement des modèles \cite{hireche_experimental_2020}.\bigskip

\echaf{Continuer expl sur travaux de Gaëlle + redériver pour régénérateur}Dans la réalité, les ondes acoustiques en jeu dans les machines thermoacoustiques ne sont ni totalement \og à ondes stationnaires \fg{} (voir le cycle thermodynamique sur la figure~\ref{fig:CycleBrayton}) ni tout à fait \og à ondes progressives \fg{} (dont le cycle est présenté sur la figure~\ref{fig:CycleStirling}). Il a d'ailleurs été montré dans la littérature portant sur les machines à plusieurs sources qu'il existe une vitesse acoustique optimale pour une pression acoustique donnée~\cite{poignand_etude_2006}. Les puissances thermiques pompées et le gradient de températures le long du régénérateur sont les plus élevées lorsque la vitesse acoustique atteint une amplitude donnée par

\begin{subequations}
	\begin{align}
		|u|_{\sf opt} &= \sqrt{\frac{4\omega\left(\echaf{kh+k_se_s}\right)\left(1-\Prandtl^2\right)\left(\echaf{1-\frac{\delta_\nu}{h}+\frac{\delta_\nu^2}{2h^2}}\right)}{\delta_\kappa \rho_0 C_p \left(1-\Prandtl\sqrt{\Prandtl}\right)}},	\label{eq:u_mag_opt_Gaelle}\\
		\intertext{et un déphasage écrit}
		\angle u_{\sf opt} &= \arctan\left[-\frac{1+\sqrt{\Prandtl}-\frac{\delta_\nu}{h}}{1-\sqrt{\Prandtl}+\frac{\delta_\nu}{h}\sqrt{\Prandtl}}\right].	\label{eq:u_arg_opt_Gaelle}
	\end{align}
	\label{eq:u_opt_Gaelle}
\end{subequations}
\echaf{Remplacer $kh+k_s e_s$ par $\Phi k + (1-\Phi)k_s$ ?}

Cette machine nécessite entre autres choses le respect de la condition $\delta_{\kappa,\nu} \gg r_h$, de sorte à avoir un excellent contact thermique entre le fluide et le solide poreux. Pour le fluide considéré, les épaisseurs de couches limites sont tracées en fonction de la fréquence et comparé au rayon hydraulique sur la figure~\ref{fig:dKdV}.

\begin{figure}[!ht]
    \centering
    \external{fig_dKdV}
%    \externalremake
    \begin{tikzpicture}
    \def\width{.9*\textwidth};
    \def\height{.45*\width};
    \def\spx{.25cm};
    \def\spy{1.25cm};
    \def\legx{.5cm};
    \def\legy{\legx};
    \def\prop{.45};
    \def\xcursor{47};
    \def\ycursorV{9.112e-5};
    \def\ycursorK{1.4452e-4};
    \def\rh{2.81e-5};
    
    \begin{axis}[name=dKdV,width={\width},height={\height},
    grid=both, minor tick num=10, 
    grid style={line width=.1pt, draw=gray!10},
    major grid style={line width=.2pt,draw=gray!50},
    xlabel={Fréquence $f$ (\unit{\Hz})},
    ylabel={\'Epaisseurs de couches limites $\delta_{\kappa,\nu}$ (\unit{\m})},
    xmin=0,xmax=75,ymin=0,ymax=.5/1000,
    xtick={0,25,50,75,100},
    extra x ticks={47},
    extra x tick style={
        grid=major,
        xticklabel={\num{47}},
        xticklabel style={yshift=0, anchor=north}
        },
    extra y ticks={\rh},
    extra y tick style={
        grid=major,
        yticklabel={$r_h$},
        yticklabel style={yshift=1mm, anchor=east}
        },
    ytick={0,1e-4,...,10e-4},
%    ytick={0,2.81/100000,9.1120/100000,1.4452/10000,
%    	2.5/10000,5/10000,1/1000},
    scaled y ticks = false,
    domain=0:100,
    legend cell align={left},
    legend style = {at={($(1,1)+(-2mm,-2mm)$)},anchor = north east,rounded corners}
    ]
        \addplot[solid,ultra thick,draw=Plasma1] file {../fig/fig_dKdV/data/data_dK.txt};
        \addplot[solid,ultra thick,draw=Plasma64] file {../fig/fig_dKdV/data/data_dV.txt};
        \filldraw[Plasma1] (\xcursor,\ycursorK) circle (2pt) node[above right]{$\delta_\kappa=\qty{\ycursorK}{\meter}$};
        \filldraw[Plasma64] (\xcursor,\ycursorV) circle (2pt) node[below left]{$\delta_\nu=\qty{\ycursorV}{\meter}$};
%        \draw[dashed,black!50] (\xcursor,0) -- (\xcursor,\ycursorK);
%        \draw[dashed,Plasma33] (\xcursor,\ycursorK) -- (0,\ycursorK) node[left]{\num{1.4452e-4}};
%        \draw[dashed,Plasma66] (\xcursor,\ycursorV) -- (0,\ycursorV) node[left]{\num{9.1120e-5}};
%         \draw[dashed,black!50] ({axis cs:\xcursor,0}|-{rel axis cs:0,0}) -- ({axis cs:\xcursor,0}|-{rel axis cs:0,\ycursorK});
       \addplot[loosely dashed,draw=black, ultra thick] {\rh};
       \draw(0,\rh) node[above right]{\qty{\rh}{\meter}};
        
        \legend{$\delta_\kappa$ \\ $\delta_\nu$ \\ $r_h$ \\};
    \end{axis}
\end{tikzpicture}
    \caption{\'Evolution des épaisseurs de couches limites thermique $\delta_\kappa$ et visqueuse $\delta_\nu$ en fonction de la fréquence, définies par le système d'équations~\eqref{eq:CouchesLimites}. Elles sont comparées au rayon hydraulique $r_h$.}
    \label{fig:dKdV}
\end{figure}

Les dimensions et paramètres du régénérateur sont résumés dans le tableau \ref{tab:ParamHydrauTAC}.

\begin{table}[!ht]
    \caption{Paramètres hydrauliques du régénérateur à la fréquence de fonctionnement, \linebreak $f=\qty{47}{\Hz}$}
    \label{tab:ParamHydrauTAC}
    \centering
    \begin{tabular}{l@{\hspace{1cm}}l}
    	\hline
    	\textbf{Paramètre [unité]} & \textbf{Valeur} \\\hline\hline
    	Diamètre du noyau $D_{\sf reg}$ [\unit{\meter}] & \num{148e-3} \\
    	Longueur du régénérateur $L_{\sf reg}$ [\unit{\meter}] & \num{39e-3} \\
    	Diamètre du fil $d_w$ [\unit{\meter}] & \num{53e-6} \\
        Rayon hydraulique $r_h$ [\unit{\meter}] & \num{2.81e-5} \\
        Porosité du noyau $\Phi$ [\unit{\percent}] & \num{68}\\
        Couche limite thermique $\delta_\kappa$ [\unit{\meter}] & \num{1.4452e-4} \\
        Couche limite visqueuse $\delta_\nu$ [\unit{\meter}] & \num{9.1120e-5} \\
        \echaf{Perméabilité} [\unit{\meter\squared}] & \num{2.68e-10} \\
        \hline
    \end{tabular}
\end{table}

\paragraph{\'Echangeurs de chaleur}
Le pompage de chaleur par effet thermoacoustique est exploité par des échangeurs, conçus avec le reste du dispositif expérimental en utilisant le logiciel \textsc{DeltaEC}.  Leur longueur est choisie 

%\medskip


\subparagraph{L'échangeur ambiant} est fabriqué en 

En effet, dans l'application du \textsc{Tacot}, il est nécessaire d'empêcher l'échauffement global de la machine, puisque le pompage de chaleur continue tant que de l'énergie mécanique est apportée au système par les sources acoustiques. 
%\medskip

\subparagraph{L'échangeur froid}
De plus, une pompe à chaleur retire par principe de la chaleur à une source à refroidir, dite \og source froide \fg{}. Toutefois, cet échangeur remplit une autre fonction dans le cadre de ce projet. La caractérisation des performances de la pompe à chaleur requiert la connaissance du flux de chaleur retiré à la source froide.

\subsection{Instrumentation}
L'instrumentation utilisée est basée sur celle conçue au début du projet \cite{ramadan_design_2021}, tout en modifiant quelques éléments.

%\bigskip

\subsubsection{Banc de mesure ?}
\echaf{photo du banc}

\subsubsection{Chaîne d'excitation}
La chaîne d'excitation est présentée. Elle est assez simple, et se compose d'un générateur de fonction à deux canaux (TekTronix AFG3022). Chaque canal est ensuite connecté à un amplificateur pour chaque source acoustique. La source principale (RIX Industries, 1S241M) est alimentée par un amplificateur QSC PLD4.5, et la source secondaire (Peerless, GBS135F) par un amplificateur Yamaha P3500S.\medskip


\subsubsection{Chaîne d'acquisition}
La chaîne d'acquisition se compose de plus de trente capteurs. Tous ne sont pas utilisés, mais peuvent servir de contrôle durant une expérience, pour s'assurer du bon déroulement de celle-ci.

\paragraph{Alimentation électrique des sources} L'alimentation électrique de la source acoustique principale est mesurée au moyen d'une sonde différentielle pour la tension \echaf{et le courant ?}. Pour la source acoustique secondaire, un multimètre et une pince de courant se chargent de mesurer sa consommation électrique. En parallèle, les tensions aux bornes des deux sources sont affichées sur un oscilloscope pour s'assurer de leur déphasage.

\paragraph{Température} Dix-neuf thermocouples Type K de \qty{.5}{\milli\meter} de diamètre sont placés de la manière suivante : quinze thermocouples mesurent la température en différentes positions du noyau, un devant la source acoustique principale, deux derrière celle-ci, et un derrière la source acoustique secondaire. Cependant, la carte d'acquisition utilisée (National Instruments, NI9213) ne comporte que seize entrées, il \echaf{convient} donc, suivant les informations recherchées dans une expérimentation donnée, de sélectionner les trois thermocouples dont les signaux sont mis de côté. Dans tous les résultats de mesures discutés dans la suite, les thermocouples du noyau et de devant la source acoustique principale sont connectés. Le placement de ces thermocouples d'intérêt est représenté sur la figure~\ref{fig:TCdansNoyau} par les symboles `\textcolor{cyan}{\textbullet}'.

\begin{figure}[!ht]
    \centering
    \external{fig_TCdansNoyau}
    %\externalremake
    \begin{tikzpicture}[scale=.2]
	\def\rCHX{14cm};
	\def\lCHX{.7cm};
	\def\rREG{14.8cm};
	\def\lREG{3.9cm};
	\def\rAHX{11cm};
	\def\lAHX{2.3cm};
	
	
	\fill[pattern=horizontal lines,pattern color=MatlabOrange,draw=black] (0,-\rAHX) rectangle ++(\lAHX,2*\rAHX);
	\draw[MatlabOrange] (0,-\rAHX) node(AHX)[below left]{\'Echangeur ambiant};
	\foreach \r in {-.9,0,.9}{
		\draw[cyan] (-.1*\lREG,\r*\rAHX) node{\textbullet};
	}
	\filldraw[draw=black,fill=gray!50!white] (0,\rAHX) rectangle (\lAHX,\rREG);		% côtés où l'eau circule
	\filldraw[draw=black,fill=gray!50!white] (0,-\rAHX) rectangle (\lAHX,-\rREG);	%
	
	\draw[MatlabOrange,->] (AHX.north) to[out=90,in=180] (-.1*\lCHX,-.5*\rAHX);
	
	\begin{scope}[xshift=\lAHX] % Reg
		\fill[pattern=crosshatch,pattern color=gray,draw=black] (0,-\rREG) rectangle ++(\lREG,2*\rREG);
		\draw[black] (\lREG/2,\rREG) node[above]{Régénérateur};		
		\foreach \x in {.1,.5,.9}{
			\foreach \r in {-.9,0,.9}{
				\draw[cyan] (\x*\lREG,\r*\rREG) node{\textbullet};
		}}
	\end{scope}
	
	\begin{scope}[xshift=\lAHX+\lREG] % CHX
		\fill[pattern=horizontal lines,pattern color=MatlabBlue,draw=black] (0,-\rCHX) rectangle ++(\lCHX,2*\rCHX);
		\draw[MatlabBlue] (\lCHX,-\rCHX) node(CHX)[below right]{\'Echangeur froid};
		\foreach \r in {-.9,0,.9}{
		\draw[cyan] (\lCHX+.1*\lREG,\r*\rCHX) node{\textbullet};
	}
	\filldraw[draw=black,fill=gray!50!white] (0,\rREG) rectangle (\lCHX,\rCHX);		% côtés où l'eau circule
	\filldraw[draw=black,fill=gray!50!white] (0,-\rREG) rectangle (\lCHX,-\rCHX);	%
	
	\draw[MatlabBlue,->] (CHX.north) to[out=90,in=0] (1.1*\lCHX,-.5*\rCHX);
	\end{scope}
	
	\draw[green!50!black] (0,0) node[left]{\begin{tabular}{rl}Source & \\ acoustique & $\leftarrow$ \\ secondaire &\end{tabular}};
	\draw[green!50!black] ({\lCHX+\lREG+\lAHX},0) node[right]{\begin{tabular}{rl}	
	 & Source \\ $\rightarrow$ \textcolor{cyan}{\textbullet} & acoustique \\ & principale\end{tabular}};
	
\end{tikzpicture}
    \caption{Emplacement des thermocouples dans le noyau thermoacoustique. Zoom sur l'encadré violet de la figure~\ref{fig:SchemaGeneralTACOT}.}
    \label{fig:TCdansNoyau}
\end{figure}

\paragraph{Pression dynamique} Quatre sondes piézoélectriques (PCB Piezotronics, 113B28) captent les oscillations de pression dans la pompe à chaleur. Deux sont placées à l'arrière de chacune des sources acoustiques, et les deux autres dans le canal de rétroaction de la cavité thermoacoustique. Les capteurs sont ensuite connectés à une carte d'acquisition (National Instruments, NI9234). 
%Toutefois, la longueur d'onde dans le mélange de gaz vaut $\lambda=\qty{11.7}{\meter}$ à la fréquence de fonctionnement $f=\qty{47}{\hertz}$ et est suffisamment grande pour garantir une amplitude de pression constante dans toute la machine.

\paragraph{Pression statique} Deux capteurs (Endress, Cerabar PMP21) sont connectés sur les deux tuyaux d'alimentation en gaz de la pompe  à chaleur d'un côté, et sur une carte d'acquisition (National Instruments, NI9234) de l'autre. Les arrivées de gaz se trouvent de part et d'autre de la source acoustique principale et ont pour but d'éviter une surpression sur sa face avant ou arrière et son endommagement.

\paragraph{Puissance extraite par l'échangeur ambiant} Le fonctionnement de cet échangeur est détaillé dans l'annexe~\ref{chap:AHX}. Pour déterminer la quantité de chaleur extraite du côté ambiant du noyau, la différence de température entre l'entrée d'eau de l'échangeur et sa sortie d'eau est mesurée grâce à deux sondes de platine PT100 connectées sur une carte d'acquisition (National Instruments, NI9217).

\paragraph{Déplacement des sources} Le piston de chaque source acoustique est équipé d'un accéléromètre. Pour la source acoustique principale, l'accéléromètre (MMF, KS91C) est collé sur la face arrière, tandis que pour la source secondaire, le capteur (PCB Piezotronics, 352C23) est collé sur la face avant. Ces capteurs sont choisis de sorte à ne pas trop varier la masse de l'équipage mobile, en particulier pour la source secondaire où la masse du piston et celle de l'ensemble accéléromètre et câble sont du même ordre de grandeur.\bigskip

Toutes les connexions entre l'intérieur de la machine sous haute pression statique et l'extérieur se font via des traversées étanches. Pour les capteurs, il s'agit de HF2-8CU+16K de Spectite, dimensionnées pour \qty{550}{\bar}. Pour les sources acoustiques, une traversée FA17613 de Solid Sealing Technology est choisie, et pour la source acoustique secondaire, le modèle FA36735 du même fabricant est retenu.

%Les signaux de tensions aux bornes des sources acoustiques sont acquis par une carte d'acquisition (National Instruments, (\echaf{modèle}), après connexion à une sonde de tension (\echaf{modèle}). Deux accéléromètres (\echaf{modèle}) sont collés sur les sources pour mesurer leur déplacement. Pour connaître la pression acoustique dans la cavité thermoacoustique, quatre sondes (\echaf{modèle}) sont placées respectivement à l'arrière de la source principale, à l'arrière de la source secondaire 


%\begin{figure}[!ht]
%    \centering
%    \external{fig_ChaineAcqui}
%    %\externalremake
%    \input{../fig/fig_ChaineAcqui/tex/fig_ChaineAcqui.tex}
%    \caption{Carte de la chaine d'acquisition et d'alimentation du réfrigérateur TACOT}
%    \label{fig:ChaineAcqui}
%\end{figure}

%\begin{itemize}
%    \item GBF
%    \item Amplis
%    \begin{itemize}
%        \item QSC
%        \item Yamaha
%    \end{itemize}
%    \item Sondes de tension
%    \item Cartes NI
%    \begin{itemize}
%        \item Pression statique
%        \item Pression dynamique
%        \item Thermocouples
%        \item PT100
%        \item Accéléromètres
%    \end{itemize}
%    \item LabVIEW d'acquisition
%\end{itemize}

%Pour étudier la distribution de température le long de l'axe du noyau, ainsi que dans les dimensions transverses, Seize thermocouples sont placés sur un plan et représentés par les symboles~`\textcolor{cyan}{\textbullet}' sur la figure~\ref{fig:TCdansNoyau}. Neuf sont placés au c\oe{}ur du noyau, dans le régénérateur. Trois sont fixés à l'extérieur du noyau, hors de l'échangeur ambiant, et trois autres sur l'extérieur de l'échangeur froid. Enfin, un dernier thermocouple est positionné au voisinage de la source acoustique principale, en vis-à-vis de l'échangeur froid.



%\subsection{Emplacement des capteurs}
%
%\begin{figure}[!ht]
%    \centering
%    \external{fig_ThermocouplesDefinition}
%    %\externalremake
%    \begin{tikzpicture}
    \def\LX{1};
    \def\LY{2};
    \def\CoreX{1.5};
    \def\CoreY{.9*\LY};
    
    \draw[line width=.5mm] (-2.5*\LX,0) to[out=90,in=-180] (-\LX,\LY) -- ++(2*\LX,0) -- ++(.5*\LX,-2*\LY/3) -- ++(.2*\LX,0) -- ++(0,2*\LY/3);
\draw[line width=.5mm] (\LX,\LY) -- ++(\LX,0) to[out=0,in=90] (3.5*\LX,0);

\draw[line width=.5mm] (-\LX,\CoreY) -- ++(\CoreX,0);
\draw[fill=PythonBlue] (-.9*\LX,0) -- ++(0,\CoreY) to[out=-80,in=90] (-.7*\LX,0);
\draw ({-\LX+.4*\CoreX},0) -- ++(0,\CoreY);
\draw ({-\LX+.9*\CoreX},0) -- ++(0,\CoreY);

\draw[fill=PythonBlue] (1.6*\LX,0) |- ++(.3*\LX,.9*\LY/3) |- ++(\LX,.2*\LY) arc (90:0:.05) -- ++(0,-.5*\LY);
    
    \begin{scope}[xscale=1,yscale=-1]
        \draw[line width=.5mm] (-2.5*\LX,0) to[out=90,in=-180] (-\LX,\LY) -- ++(2*\LX,0) -- ++(.5*\LX,-2*\LY/3) -- ++(.2*\LX,0) -- ++(0,2*\LY/3);
\draw[line width=.5mm] (\LX,\LY) -- ++(\LX,0) to[out=0,in=90] (3.5*\LX,0);

\draw[line width=.5mm] (-\LX,\CoreY) -- ++(\CoreX,0);
\draw[fill=PythonBlue] (-.9*\LX,0) -- ++(0,\CoreY) to[out=-80,in=90] (-.7*\LX,0);
\draw ({-\LX+.4*\CoreX},0) -- ++(0,\CoreY);
\draw ({-\LX+.9*\CoreX},0) -- ++(0,\CoreY);

\draw[fill=PythonBlue] (1.6*\LX,0) |- ++(.3*\LX,.9*\LY/3) |- ++(\LX,.2*\LY) arc (90:0:.05) -- ++(0,-.5*\LY);
    \end{scope}
    
    
    \draw[dashed,rounded corners,PythonRed] (.55,.95*\LY) rectangle ++(-.75*\CoreX,-1.9*\LY) node[midway]{\rotatebox{90}{Noyau TA}};
    
    \begin{scope}[xshift=5cm,xscale=2.5,yscale=2]
        \draw (0,0) |- ++(2,1.5) -- ++(0,-1.5);
        \draw (.5,0) -- ++(0,1.5);
        \draw (1.5,0) -- ++(0,1.5);
        \draw[line width=1mm] (-.2,1.5) -- ++(2.4,0);
    
        \begin{scope}[xscale=1,yscale=-1]
            \draw (0,0) |- ++(2,1.5) -- ++(0,-1.5);
            \draw (.5,0) -- ++(0,1.5);
            \draw (1.5,0) -- ++(0,1.5);
            \draw[line width=1mm] (-.2,1.5) -- ++(2.4,0);
        \end{scope}
        % \foreach \x [evaluate=\x] in {0,...,4}{
        %     \foreach \y [evaluate=\y] in {1,...,3}{
        %     \draw (\x,\y) node[]{$t$};}}
    \end{scope}
    
    %\draw[line width=1mm] (5*\LX,.5*\LY) -- ++(6*\LX,0);
    %\draw[line width=1mm] (5*\LX,-.5*\LY) -- ++(6*\LX,0);
    %
    %\draw (-2.5*\LX,\LY) node[above]{\bf (a)};
    %\draw (5*\LX,\LY) node[above]{\bf (b)};
    
\end{tikzpicture}
%    \caption{Emplacements des thermocouples dans le noyau thermoacoustique}
%    \label{fig:ThermocouplesDefinition}
%\end{figure}

\section{Protocole expérimental}\label{chap:ProtocolExpe}
Pour l'étude de l'influence de la gravité sur la distribution de température dans son noyau et ses performances, le réfrigérateur doit pouvoir être orienté dans toutes les orientations utiles. Pour ce faire, il est suspendu par des palans grâce aux fixations situées à ses extrémité  et au milieu dans le sens de sa longueur. La figure~\ref{fig:TACOTSuspendu_Frigo} présente le réfrigérateur accroché à ses extrémités, et la figure~\ref{fig:TACOTSuspendu_Palans} les trois palans pour le soutenir. Les deux palans de couleur grise, initialement présents pour régler l'inclinaison de la pompe à chaleur par rapport à l'axe horizontal, et le troisième de couleur bleue pour ajouter une direction de rotation autour de l'axe de symétrie. Celui-ci permet en outre de plus aisément passer d'une orientation à l'autre. 

\begin{figure}[!ht]
    \centering
	\begin{subfigure}{.45\textwidth}
		\centering
		\external{fig_SystemeAccroche_Machine}
		\tikz{\draw(0,0) node{\includegraphics[width=.9\textwidth]{../fig/fig_SystemeAccroche/Machine_horizBetter_cropped.jpg}};}
		\caption{}
		\label{fig:TACOTSuspendu_Frigo}
	\end{subfigure}		%
	\begin{subfigure}{.45\textwidth}
		\centering
		\external{fig_SystemeAccroche_Palan}
		\tikz{\draw(0,0) node{\includegraphics[width=.9\textwidth]{../fig/fig_SystemeAccroche/Palans.jpg}};}
		\caption{}
		\label{fig:TACOTSuspendu_Palans}
	\end{subfigure}	    
    \caption{Photographies \subref{fig:TACOTSuspendu_Frigo} du refrigérateur accroché et \subref{fig:TACOTSuspendu_Palans} des palans formant le système de suspension.}
    \label{fig:TACOTSuspendu}
\end{figure}

\subsection{Définition des orientations}

Les orientations choisies au moyen des palans sont décrites par deux angles $\psi_v$ et $\psi_h$. Le premier désigne l'angle entre l'axe horizontal et l'axe de symétrie du réfrigérateur, tandis que le second, la rotation autour de cet axe de symétrie. Les orientations utilisées dans les différentes parties de ce manuscript sont présentées sur les figures~\ref{fig:OrientationCore_H} et \ref{fig:OrientationCore_V}. Cette figure, dans laquelle la gravité est toujours dirigée vers le bas de la page, présente également les emplacements et les numéros d'identification des thermocouples utilisés.\medskip

\begin{figure}[!htp]
    \centering
	\begin{subfigure}[c]{.4\textwidth}
		\centering
		\external{fig_OrientationCore_H1}
%    	\externalremake
		\begin{tikzpicture}[scale=2/3]

%    \def\lenreg{2};
%    \def\diam{3};
    \def\spy{2};
    \def\xdist{8cm};
    \def\ydist{-7cm};
%    \def\persp{20};
%    
%    \def\LX{1};
%    \def\LY{2};
%    \def\CoreX{1.5};
%    \def\CoreY{.9*\LY};
%    

	\def\L{2.1};
	\def\R{5};
	\def\HX{.25};
	\def\decalage{\R/2-\L/2};
	
		
			
		\fill[right color=MatlabBlue,left color=MatlabOrange, draw=black] (\decalage,0) rectangle ++(\L,\R);
		\draw[fill=MatlabOrange] (\decalage,0) rectangle ++(-\HX,\R);
		\draw[fill=MatlabBlue] (\decalage+\L,0) rectangle ++(\HX,\R);

		\foreach \z [evaluate=\z] in {0,...,4}{
			\foreach \r [evaluate=\r as \num using int(\r+1 + 3*\z)] in {0,...,2}{
				\draw ({\decalage+.5+\L-\z*(1+\L)/4},{-(\R-.4)/2*\r+\R-.2}) node[minimum size=10pt,draw,circle,fill=white,opacity=.7,text opacity=1]{} node(n\z\r){\scriptsize \num};
}}

%		\draw (n01.east) node [right]{0 $\rightarrow$ \begin{tabular}{l}Source\\acoustique\\principale\end{tabular}};
		\draw ($(n01)+(1.5,0)$) node[minimum size=10pt,draw,circle,fill=white,opacity=.7,text opacity=1]{} node(RIX) {\scriptsize 0};% node[anchor=west]{\begin{tabular}{rl}
%		& Source\\
%		$\rightarrow$ & acoustique\\
%		& principale
%		\end{tabular}};
		\draw (n30.north west) node [above, fill=white, fill opacity=.7, text opacity=1]{\textcolor{MatlabOrange}{\textbf{Ambiant}}};
		\draw (n10.north east) node [above, fill=white, fill opacity=.7, text opacity=1]{\textcolor{MatlabBlue}{\textbf{Froid}}};
\end{tikzpicture}
		\caption{}
		\label{fig:OrientationCore_H1}
	\end{subfigure}
	\begin{subfigure}[c]{.4\textwidth}
		\centering
		\external{fig_OrientationCore_H1_Schema}
%    	\externalremake
		\tikz{\draw(0,0) node{\includegraphics[width=.9\textwidth]{../fig/fig_OrientationCore/tex/TACOT.png}};}
		\caption{}
		\label{fig:OrientationCore_H1_Schema}
	\end{subfigure} \vspace{1cm}
	
	\begin{subfigure}[c]{.4\textwidth}
		\centering
		\external{fig_OrientationCore_H2}
%    	\externalremake
		\begin{tikzpicture}[scale=2/3]

%    \def\lenreg{2};
%    \def\diam{3};
    \def\spy{2};
    \def\xdist{8cm};
    \def\ydist{-7cm};
%    \def\persp{20};
%    
%    \def\LX{1};
%    \def\LY{2};
%    \def\CoreX{1.5};
%    \def\CoreY{.9*\LY};
%    

	\def\L{2.1};
	\def\R{5};
	\def\HX{.25};
	\def\decalage{\R/2-\L/2};
	
%	\draw[opacity=0] (\decalage,0) rectangle ++(-\HX,\R); %%% Pour l'alignement vertical
	
	\begin{scope}[xslant=1,yscale=.5]
		
		\fill[shading=axis,right color=MatlabBlue,left color=MatlabOrange, shading angle=45, draw=black] (\decalage,0) rectangle ++(\L,\R);
		\draw[fill=MatlabOrange] (\decalage,0) rectangle ++(-\HX,\R);
		\draw[fill=MatlabBlue] (\decalage+\L,0) rectangle ++(\HX,\R);

		\foreach \z [evaluate=\z] in {0,...,4}{
			\foreach \r [evaluate=\r as \num using int(\r+1 + 3*\z)] in {0,...,2}{
				\draw ({\decalage+.5+\L-\z*(1+\L)/4},{-(\R-.4)/2*\r+\R-.2}) node[minimum size=10pt,draw,circle,fill=white,opacity=.7,text opacity=1]{} node(n\z\r){\scriptsize \num};
}}

%		\draw (n01.east) node [right]{$\rightarrow$ \begin{tabular}{l}Source\\acoustique\\principale\end{tabular}};
		\draw ($(n01)+(1.5,0)$) node[minimum size=10pt,draw,circle,fill=white,opacity=.7,text opacity=1]{} node {\scriptsize 0};% node[anchor=west]{\begin{tabular}{rl}
%		& Src\\
%		$\rightarrow$ & ac\\
%		& princ
%		\end{tabular}};
		\draw (n30.north west) node [above, fill=white, fill opacity=.7, text opacity=1]{\textcolor{MatlabOrange}{\textbf{Ambiant}}};
		\draw (n10.north east) node [above, fill=white, fill opacity=.7, text opacity=1]{\textcolor{MatlabBlue}{\textbf{Froid}}};
	\end{scope}
	
	
		
\end{tikzpicture}
		\caption{}
		\label{fig:OrientationCore_H2}
	\end{subfigure} %\vspace{1cm}
	\begin{subfigure}[c]{.4\textwidth}
		\centering
		\external{fig_OrientationCore_H2_Schema}
%    	\externalremake
		\tikz{\draw(0,0) node{\includegraphics[width=.9\textwidth]{../fig/fig_OrientationCore/tex/TACOT.png}};}
		\caption{}
		\label{fig:OrientationCore_H2_Schema}
	\end{subfigure} 
	\caption{Orientations horizontales du réfrigérateur thermoacoustique avec les positions des thermocouples et leurs numéro. Pour chaque cas, la gravité est orientée vers le bas de la page. \subref{fig:OrientationCore_H1} et \subref{fig:OrientationCore_H1_Schema} orientation `\texttt{H1}' ; \subref{fig:OrientationCore_H2} et \subref{fig:OrientationCore_H2_Schema} orientation `\texttt{H2}'}
	 \label{fig:OrientationCore_H}
\end{figure}

\begin{figure}[!htp]
	\centering	
	\begin{subfigure}[c]{.4\textwidth}
		\centering
		\external{fig_OrientationCore_V1}
%    	\externalremake
		%\fill[top color=red!25, bottom color=blue!25, draw=black] (0,0) rectangle ++(\R,\L);
%\draw[fill=blue!25] (0,0) rectangle ++(\R,-\HX);
%\draw[fill=red!25] (0,\L) rectangle ++(\R,\HX);
%
%\foreach \z [evaluate=\z] in {0,...,4}{
%	\foreach \r [evaluate=\r as \num using int(\r+1 + 3*\z)] in {0,...,2}{
%		\draw ({-(\R-.4)/2*\r+\R-.2},{\z*(1+\L)/4-.5}) node(n\z\r){\num};
%}}
%
%\draw (n40.south east) node [right]{AHX};
%\draw (n00.north east) node[right]{CHX};
%\draw (n01.south) node [below]{\shortstack{ $\downarrow$ \\Source acoustique principale}};
%
%\draw (0,\L+2*\HX+\spy) node [anchor=west]{\textbf{(c)} \texttt{V1}};

\begin{tikzpicture}[scale=2/3]

%    \def\lenreg{2};
%    \def\diam{3};
    \def\spy{2};
    \def\xdist{8cm};
    \def\ydist{-7cm};
%    \def\persp{20};
%    
%    \def\LX{1};
%    \def\LY{2};
%    \def\CoreX{1.5};
%    \def\CoreY{.9*\LY};
%    

	\def\L{2};
	\def\R{5};
	\def\HX{.35};
	\def\decalage{\R/2-\L/2};
	
	\begin{scope}[yslant=tan(22.5)]	
		
		\node at (0,-\HX) (NewO) {};
	
		\fill[shading=axis,right color=MatlabBlue,left color=MatlabOrange, shading angle=22.5, draw=black] (0,0) rectangle ++(\R,\L);
		\draw[fill=MatlabBlue] (0,0) rectangle ++(\R,-\HX);
		\draw[fill=MatlabOrange] (0,\L) rectangle ++(\R,\HX);

		\foreach \z [evaluate=\z] in {0,...,4}{
			\foreach \r [evaluate=\r as \num using int(\r+1 + 3*\z)] in {0,...,2}{
				\draw ({-(\R-.4)/2*\r+\R-.2},{\z*(1+\L)/4-.5}) node[minimum size=10pt,draw,circle,fill=white,opacity=.7,text opacity=1]{} node(n\z\r){\scriptsize \num};
}}

%		\draw (n40.south east) node [right, fill=white, fill opacity=0, text opacity=1]{\textcolor{MatlabOrange}{\textbf{Ambiant}}};
%		\draw (n00.north east) node[right, fill=white, fill opacity=0, text opacity=1]{\textcolor{MatlabBlue}{\textbf{Froid}}};
		\draw ($(n01.south)+(0,-1.1)$) node[minimum size=10pt,draw,circle,fill=white,opacity=.7,text opacity=1]{} node (RIX){\scriptsize 0};% node[anchor=north]{\begin{tabular}{c}
%		$\downarrow$\\
%		Source acoustique principale
%		\end{tabular}};

	\end{scope}
	\begin{pgfonlayer}{background}
		\draw[->, very thick] (NewO.center) -- ++(22.5:1.2*\R) node [above] {$\mathbf e_{y,0}$};
		\draw[->, very thick] (NewO.center) -- ++(90:1.2*\R) node [left] {$\mathbf e_{z,0}$};
		\draw[->, very thick] (NewO.center) -- ++(-45:1.2*\R) node [right] {$\mathbf e_{x,0}$};
  	\end{pgfonlayer}		
\end{tikzpicture}
		\caption{}
		\label{fig:OrientationCore_V1}
	\end{subfigure} 
	\begin{subfigure}[c]{.4\textwidth}
		\centering
		\external{fig_OrientationCore_V2}
%    	\externalremake
		%\fill[top color=blue!25, bottom color=red!25, draw=black] (0,0) rectangle ++(\R,\L);
%\draw[fill=red!25] (0,0) rectangle ++(\R,-\HX);
%\draw[fill=blue!25] (0,\L) rectangle ++(\R,\HX);
%
%\foreach \z [evaluate=\z] in {0,...,4}{
%	\foreach \r [evaluate=\r as \num using int(\r+1 + 3*\z)] in {0,...,2}{
%		\draw ({(\R-.4)/2*\r+.2},{-\z*(1+\L)/4+\L+.5}) node(n\z\r){\num};
%}}
%
%\draw (n01.north) node [above]{\shortstack{Source acoustique principale\\ $\uparrow$}};
%\draw (n42.north east) node [right]{AHX};
%\draw (n02.south east) node [right]{CHX};
%
%\draw (0,\L+2*\HX+\spy) node [anchor=west]{\textbf{(d)} \texttt{V2}};

\begin{tikzpicture}[scale=2/3]

%    \def\lenreg{2};
%    \def\diam{3};
    \def\spy{2};
    \def\xdist{8cm};
    \def\ydist{-7cm};
%    \def\persp{20};
%    
%    \def\LX{1};
%    \def\LY{2};
%    \def\CoreX{1.5};
%    \def\CoreY{.9*\LY};
%    

	\def\L{1.5};
	\def\R{5};
	\def\HX{.25};
	\def\decalage{\R/2-\L/2};

		\fill[top color=blue!25, bottom color=red!25, draw=black] (0,0) rectangle ++(\R,\L);
		\draw[fill=red!25] (0,0) rectangle ++(\R,-\HX);
		\draw[fill=blue!25] (0,\L) rectangle ++(\R,\HX);

		\foreach \z [evaluate=\z] in {0,...,4}{
			\foreach \r [evaluate=\r as \num using int(\r+1 + 3*\z)] in {0,...,2}{
				\draw ({(\R-.4)/2*\r+.2},{-\z*(1+\L)/4+\L+.5}) node[minimum size=10pt,draw,circle,fill=white,opacity=.7,text opacity=1]{} node(n\z\r){\scriptsize \num};
}}

		\draw ($(n01.north)+(0,.5cm)$) node[minimum size=10pt,draw,circle,fill=white,opacity=.7,text opacity=1]{} node(RIX){\scriptsize 0} node[anchor=south]{\begin{tabular}{c}
		Source acoustique principale\\
		$\uparrow$
		\end{tabular}};
		\draw (n42.north east) node [right]{Ambiant};
		\draw (n02.south east) node [right]{Froid};
%		\draw (n41.south) node [below]{\textcolor{white}{\shortstack{Source acoustique principale\\ $\uparrow$}}};
		
\end{tikzpicture}
		\caption{}
		\label{fig:OrientationCore_V2}
	\end{subfigure}  \vspace{1cm}

	\begin{subfigure}[c]{.4\textwidth}
		\centering
		\external{fig_OrientationCore_V1_Schema}
%    	\externalremake
		\tikz{\draw(0,0) node[rotate=-90]{\includegraphics[width=.9\textwidth]{../fig/fig_OrientationCore/tex/TACOT.png}};}
		\caption{}
		\label{fig:OrientationCore_V1_Schema}
	\end{subfigure} 
	\begin{subfigure}[c]{.4\textwidth}
		\centering
		\external{fig_OrientationCore_V2_Schema}
%    	\externalremake
		\tikz{\draw(0,0) node[rotate=90]{\includegraphics[width=.9\textwidth]{../fig/fig_OrientationCore/tex/TACOT.png}};}
		\caption{}
		\label{fig:OrientationCore_V2_Schema}
	\end{subfigure} 	
 	\caption{Orientations verticales du réfrigérateur thermoacoustique avec les positions des thermocouples et leurs numéro. Pour chaque cas, la gravité est orientée vers le bas de la page. \subref{fig:OrientationCore_V1} et \subref{fig:OrientationCore_V1_Schema} orientation `\texttt{V1}' ; \subref{fig:OrientationCore_V2} et \subref{fig:OrientationCore_V2_Schema} orientation  `\texttt{V2}'.}%\textcolor{red}{CHX et AHX OK ou éch. froid et éch. chaud ? + $\psi_i$ dans la caption ou la figure ?}}
    \label{fig:OrientationCore_V} %
\end{figure}



La première orientation, nommée `\texttt{H1}' et représentée sur la figure~\ref{fig:OrientationCore_H1}, est la même que dans l'article dédié à la conception du réfrigérateur \cite{ramadan_design_2021}. Dans cette configuration, le \textsc{Tacot} est placé à l'horizontale comme sur la figure~\ref{fig:TACOTSuspendu_Frigo}, et les thermocouples sont placés sur un plan vertical coplanaire à la gravité. Cette orientation est celle avec laquelle tous les résultats ont été obtenus avant le démarrage de la thèse et fait donc office de référence des orientations, soit $\psi_v=\psi_h=\qty{0}{\degree}$.\smallskip

Ensuite, la deuxième orientation est représentée sur la figure~\ref{fig:OrientationCore_H2}. Dans ce cas, référérencé en tant que `\texttt{H2}', le réfrigérateur est toujours à l'horizontale ($\psi_v=\qty{0}{\degree}$), mais pivoté autour de son axe pour placer les thermocouples sur un plan horizontal auquel la gravité est orthogonale ($\psi_h=\qty{90}{\degree}$).\smallskip

L'orientation `\texttt{V1}' est affichée sur la figure~\ref{fig:OrientationCore_V1}. Cette configuration est radicalement différentes des deux précédentes : l'axe de symétrie du réfrigérateur est vertical, avec l'échangeur froid sous l'échangeur ambiant, soit $\psi_v=\qty{-90}{\degree}$. \smallskip

Enfin, l'orientation `\texttt{V2}' affichée sur la figure~\ref{fig:OrientationCore_V2} est l'orientation inverse de la précédente. L'axe de symétrie du réfrigérateur est encore vertical, mais la source acoustique principale est cette fois au dessus du noyau thermoacoustique et $\psi_v=\qty{+90}{\degree}$.

\subsection{Acquisitions}
Les acquisitions sont réalisées en plusieurs temps. Tout d'abord et pour toutes les expériences,  l'état initial de toutes les grandeurs est acquis sur une minute et sauvegardé sous un label `\texttt{init}' à chaque début de journée de campagne. Cela permet de garder en mémoire toutes les conditions expérimentales initiales dont les valeurs peuvent potentiellement influer sur le comportement du réfrigérateur, comme par exemple la température ambiante ou la pression statique. \medskip

Ensuite, en prévision de la mesure de flux de chaleur $\dot Q_a$ extrait par l'échangeur ambiant (voir l'annexe~\ref{chap:AHX}), l'eau est préalablement mise en circulation dans cet échangeur après avoir démarré une acquisition des 30 capteurs jusqu'à stabilisation de la distribution de température dans le noyau. L'acquisition est ensuite interrompue et enregistrée avec un label `\texttt{Water}'. \bigskip

L'étape suivante dépend du type d'expérience menée : les mesures peuvent être sans ou avec acoustique, et ce, pour  différentes amplitudes de pression oscillante. \echaf{la suite part en intro sur ce qui existe déjà} En revanche, certains des paramètres d'excitation restent constants pour toutes les expériences :  le gaz est également le même dans toutes les expériences. Il est composé de \qty{65}{\percent} d'hélium et de \qty{35}{\percent} d'argon, car dans ces proportions le nombre de Prandtl est minimum \cite{belcher_working_1999} ; ce mélange est ensuite pressurisé à \qty{40}{\bar}. Dans le cas des expériences avec acoustique, le modèle \textsc{DeltaEC} prédit les meilleurs performances à la fréquence $f=\qty{47}{\hertz}$, c'est-à-dire la fréquence de résonance du système. C'est par ailleurs le seul point de fonctionnement où l'impédance électrique est supérieure à la limite basse admise par l'amplificateur, soit \qty{2}{\ohm}. Ensuite, le déphasage inter-source $\varphi_{2-1}$ est également fixé à \ang{-60} pour toutes les expériences, également indiqué comme déphasage optimale par les simulations et que des expériences préliminaires confirment.

\subsubsection{Mesures sans acoustique}\label{chap:MesureSansAcou}
Pour ces mesures de type `\texttt{heat\_{}only}', la charge thermique est appliquée au noyau sans alimenter les sources acoustiques. Cette charge thermique consiste en l'alimentation électrique de cartouches chauffantes contenues dans l'échangeur froid par une puissance connue, tandis qu'un débit d'eau de \qty{7}{\litre\per\minute} s'écoule dans l'échangeur ambiant qui se trouve de l'autre côté du noyau. 

Ces mesures doivent permettre d'étudier la distribution de température en l'absence d'écoulement oscillant, ainsi que de calculer les valeurs de conductivité thermique $k_x$ et $k_r$ ou les coefficients de pertes latérales $h_x$ et $h_r$.\medskip

Dans ce type d'expériences, les noms des zones \og froide \fg{} et \og ambiante \fg{} sont conservés pour des raisons de cohérence avec les schémas présentés auparavant, mais l'eau circulant dans l'échangeur ambiant et les cartouches chauffantes se trouvant dans l'échangeur froid, la direction du gradient de température dans le noyau thermoacoustique est inversée par rapport aux expériences avec acoustique. %Toutefois, les ordres de grandeur des différences de température sont les mêmes que pour les expériences avec acoustique, c'est-à-dire 

%Pour garder un moyen de comparaison avec les mesures avec acoustique, le mélange de gaz est le même.

\subsubsection{Mesures avec acoustique}\label{chap:MesureAvecAcou} 
Une acquisition étiquetée `\texttt{Acou}' est démarrée, puis les sources sont alimentées jusqu'à l'amplitude souhaitée. Au bout d'une heure, l'acquisition est arrêtée et sauvegardée. En l'absence d'expérience avec charge thermique, c'est la fin de l'expérience : toutes les sources acoustiques et circulations d'eau sont progressivement arrêtées et le réfrigérateur est laissé pour un retour à l'état initial.

Au cours de cette étude, trois amplitudes acoustiques sont choisies. La première correspond à un \textit{drive ratio} $DR=\frac{p_1}{p_0}=\qty{.4}{\percent}$, soit une amplitude très faible où l'effet thermoacoustique est à peine visible -- soit une différence de température de l'ordre de \qty{5}{\kelvin}. Ainsi, l'hypothèse concernant la linéarité acoustique est mieux vérifiée et peut \textit{a priori} être plus aisément comparé à la théorie linéaire. À l'inverse, le \textit{drive ratio} de la deuxième amplitude est le plus élevé avec $DR=\qty{3.5}{\percent}$, et est celui pour lequel les performances du réfrigérateur ($COP$, $Q_f$, ...) sont les plus élevées obtenues avec cette machine \cite{ramadan_design_2021}, mais aussi qui présentent de forts écarts à la théorie. La troisième est choisie à un \textit{drive ratio} intermédiaire où $DR=\qty{2}{\percent}$. \medskip

Ensuite, une charge thermique peut être appliquée au noyau dans le cadre d'une expérience annotée `\texttt{Qc\_x\_W}' où \texttt{x} représente la puissance injectée. Ce type d'expérience se fait en alimentant avec un transformateur les six cartouches chauffantes connectées en parallèle et contenues dans l'échangeur froid (voir la figure~\echaf{fig noyau}. Pour une expérience donnée, une puissance thermique est choisie selon la relation

\begin{equation}
	Q_f = \frac{E^2}{R},
	\label{eq:Qf_définitionEsurR}
\end{equation}
où $E$ est la tension appliquée aux cartouches et $R=\qty{22.4}{\ohm}$ la résistance des cartouches en parallèle. \echaf{à continuer}

\section{Post-traitement des données}
À la fin des acquisitions, les données sont post-traitées en utilisant un code \textsc{Matlab}\textss\textregistered maison, disponible sur \url{https://github.com/ohmenkoler/tacotpostprocessing}.

\subsection{Allègement des fichiers}
Pendant chaque campagne de mesure réalisée suivant le protocole \echaf{section}, des fichiers bruts de mesures sont générés. Ils présentent deux inconvénients majeurs : leur format est \texttt{*.tdms}, celui de NI LabVIEW, et ils sont très lourds. La première étape de traitement des données est d'extraire uniquement les informations nécessaires pour les préparer à d'autres actions en utilisant un script \textsc{Matlab}\textss\textregistered, et à les sauvegarder dans un fichier \texttt{*.mat}.\medskip

Pour ce faire, un fichier \texttt{*.tdms} à traiter est tout d'abord chargé. Il peut alors être tronqué à un temps $t_{fin}$ souhaité pour garantir des fichiers temporels de même longueur\footnote{Sauf si un problème d'acquisition coupe l'enregistrement tout seul\dots}. Un fichier de configuration est créé, comprenant notamment l'amplitude acoustique, la tension appliquée aux cartouches de l'échangeur froid, ou l'orientation, pour ne pas perdre ces informations après ce pré-traitement des données.\smallskip

Ensuite, la valeur initiales est retirée de chaque signal temporel de thermocouple. Cette valeur initiale est calculé sur les \qty{10}{\second} au début des signaux temporels, soit un nombre de points égal à $10 \times f_{e}$ où $f_{e}$ est la fréquence d'échantillonage qui vaut \qty{1651}{\hertz}.\smallskip

Cette fréquence est imposée par les cartes d'acquisition de pression dynamique à tout le \texttt{vi} LabVIEW. C'est bien trop élevé, aussi bien pour les quantités non oscillantes comme la température, que les données oscillant à la fréquence $f_1=\qty{47}{\hertz}$ comme la pression dynamique ou l'accélération. Il est donc nécessaire d'alléger les données, et les températures sont ré-échantillonées à une fréquence $f_{re}=\qty{.1}{\hertz}$. De leur côté, les autres données oscillantes sont coupées pour ne conserver que les \num{20} dernières périodes entières des signaux, soit un nombre de points donné par $20 \times \nicefrac{f_e}{f_1}$.\smallskip

Le cas du flux de chaleur extrait par l'échangeur ambiant $\dot Q_a$ est alors traité : pour les expériences en deux étapes ou plus (une expérience de type `\texttt{water\_only}' suivie d'une autre de type `\texttt{acou}' par exemple), il faut d'abord extraire les informations sur l'état du système avant le démarrage des sources, et en particulier l'écart de températures mesuré par les sondes de platine à l'entrée et à la sortie de l'échangeur ambiant. Cet écart peut être non-nul malgré l'atteinte d'un équilibre thermique après l'ouverture de la circulation d'eau et une attente suffisamment longue. Une première valeur de $\dot Q_a^{(wo)}$ est donc calculée et enregistrée dans un fichier \texttt{*.mat}. Lors du traitement d'un fichier avec acoustique, il est alors possible de recharger cette valeur et la soustraire à la nouvelle valeur de $\dot Q_a^{(ac)}$ calculée.\smallskip

Les amplitudes et phases des pressions dynamiques et accélérations sont calculées en utilisant le principe de la détection synchrone. Cette méthode est appliquée pour les quatre capteurs de pression dynamique et les deux accéléromètres, et en considérant cinq harmoniques.

%\echaf{nécessaire ?} Cette méthode repose sur l'orthogonalité des fonctions trigonométriques : le signal dont l'amplitude à une fréquence donnée $f_{osc}$ est recherchée est multiplié par un oscillateur à la même fréquence $f_{osc}$, puis moyenné sur une période $t_{osc}=\nicefrac{1}{f_{osc}}$ avec un filtre intégrateur. Le résultat de cette moyenne est différent de zéro dans le seul cas où la fréquence du signal $f_1$ est égale à celle de l'oscillateur $f_{osc}$.




\subsection{Tracé de figures pour analyse}

\begin{comment}
\subsubsection{Paramètres d'acquisition}
Comme dit précédemment, la fréquence de fonctionnement du \textsc{Tacot} est de \qty{47}{\hertz}, ce qui implique une fréquence d'échantillonnage au moins deux fois supérieure. Cependant, les cartes d'acquisition sont rassemblées sur une baie d'instrumentation, contraignent la fréquence d'échantillonage utilisée. Celles concernant les mesures de quantité oscillantes (pression acoustique, accélération \echaf{à vérifier}) imposent que la fréquence d'échantillonage $f_s$ soit au moins égale à \qty{1651}{\Hz}\footnote{Les acquisitions des \num{30} capteurs durent \qty{1}{\hour}, et les données sont encodées sur \qty{32}{\bit} flottants. Au total, chaque acquisition pèse \qty{713}{\mega\byte}, taille à laquelle il faut ajouter quelques \unit{\mega\byte} pour le protocole \texttt{tdms} et l'entête contenant les informations de mesure.}.
\end{comment}


\section{Conclusion}
Ce chapitre présente le réfrigérateur existant déjà au début de ce travail de thèse. L'utilisation d'une géométrie coaxiale et de deux sources est rappelée, justifiée par la nécessité de compacité de la machine.

Quelques résultats obtenus dans la littérature sont présentés, en gardant à l'esprit les écarts avec le modèle \textsc{DeltaEC}. De tous les phénomènes qui prennent place dans un tel système, la convection naturelle est l'effet qui est étudié ici. Pour cela, le système doit pouvoir être orienté dans toutes les orientations, parmi lesquelles quatre sont choisies pour leur caractère académique.

L'instrumentation modifiée pour permettre l'excitation et l'observation du réfrigérateur est décrite, en notant toutefois les limites du dispositif causées par la compacité du prototype.

Le protocole de mesure est également présenté. Il est dépendant des résultats à extraire, et est le résultat d'un grand nombre de campagnes d'expériences. Ce protocole permet d'obtenir des résultats comparables entre les différentes configurations.




