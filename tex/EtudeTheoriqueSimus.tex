\chapter{\'Etudes théoriques et simulations}
\vfill
\hrule \vspace{.5cm}
{\hypersetup{linkcolor = black}
\localtableofcontents
}%
\vspace{.5cm} \hrule
\vfill
\clearpage

\section{Détermination des coefficients empiriques du modèle transitoire}

Le modèle transitoire présenté au chapitre §\ref{chap:ModeleTransoitoire_SansConvNat} comporte des paramètres empiriques à adapter pour approcher les résultats expérimentaux. Parmi eux, il y la source de chaleur liées à la vorticité $Q_{\sf vort}$, et les coefficients d'échanges aux interfaces en $x=0$, $x=L_{\sf reg}$ et $r=R_{\sf reg}$, notés respectivement $h_{x}|_{x=0}$, $h_{x}|_{x=L_{\sf reg}}$ et $h_{r}|_{r=R_{\sf reg}}$. Ces trois derniers sont supposés uniquement dépendants de l'écart de température $\tau$, et pour des raisons de simplicité sont estimés au moyen des mesures `\texttt{heat\_only}', ce qui permet en outre de s'affranchir de la vorticité. Le modèle est alors modifié pour calculer la température du système à $p=0$ et $u=0$.