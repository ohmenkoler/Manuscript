\chapter{Index des notations}\label{chap:IndexNotations}%\addcontentsline{toc}{chapter}{\nameref{chap:IndexNotations}}

\begin{center}
    \begin{longtable}{p{.1\textwidth} p{.1\textwidth} p{.7\textwidth}}
%    	\hline
        \multicolumn{3}{c}{Lettres latines}  \\\hline
        \textbf{Symbole} & \textbf{Unité} & \textbf{Définition} \\\hline\hline \endfirsthead
        \multicolumn{3}{c}{Lettres latines (suite)}  \\\hline
        \textbf{Symbole} & \textbf{Unité} & \textbf{Définition} \\\hline\hline \endhead
		
		\hline
		\multicolumn{3}{r}{Continué sur la page suivante...} \endfoot
        \hline \endlastfoot
        $c_0$ & \unit{\meter\per\second} & Célérité du son dans le milieu \\
        $C_p$ & \unit{\joule\per\kelvin} & Capacité thermique isobare du gaz \\
        $C_s$ & \unit{\joule\per\kelvin} & Capacité thermique du solide poreux \\
        $C_v$ & \unit{\joule\per\kelvin} & Capacité thermique isochore du gaz \\
        $\COP$ & --- & Coefficient de performance, $\COP=\frac{Q_f}{W_e}$ \\
        $\COP_{\sf carnot}$ & --- & Coefficient de performance de Carnot, $\COP_{\sf carnot} = \frac{T_f}{T_c-T_f}$ \\
        $DR$ & \unit{\percent} & \textit{Drive ratio}, $DR=\frac{p_1}{p_0}$\\
        $\mathbf{g}$ & \unit{\meter\per\square\second} & Accélération terrestre, $\mathbf{g}=-g \mathbf e_{z,0}$ avec $g=\qty{9.81}{\meter\per\second\squared}$ \\
        $G$ & \echaf{?/m} & Gain thermoacoustique unitaire \\
        $\Grashof$ & --- & Nombre de Grashof \\
        $k_0$ & \unit{\radian\per\meter} & Nombre d'onde sans perte \\
        $k$ & \unit{\joule\per\meter\per\kelvin} & Conductivité thermique \\
        $K_p$ & \unit{\meter\squared} & Perméabilité hydraulique d'un matériau poreux \\
        $L_{\sf reg}$ & \unit{\meter} & Longueur du régénérateur, $L_{\sf reg}=\qty{39}{\milli\meter}$ \\
        $\Nusselt$ & --- & Nombre de Nusselt \\
        $p_0$ & \unit{\pascal} & Pression statique dans le réfrigérateur \\
        $p_1$ & \unit{\pascal} & Pression acoustique \\
        $\Peclet$ & --- & Nombre de Péclet \\
        $\Prandtl$ & --- & Nombre de Prandtl \\
        $\mathbf{e}_r$ & \echaf{---} & Vecteur unitaire transverse du noyau \\
        $R_{\sf reg}$ & \unit{\meter} & Rayon du régénérateur, $R_{\sf reg}=\qty{74}{\milli\meter}$ \\
        $\Rayleigh$ & --- & Nombre de Rayleigh \\
        $r_h$ & \unit{\meter} & Rayon hydraulique des pores du régénérateur \\
        $T_0$ & \unit{\kelvin} & Température moyenne locale du gaz dans le noyau \\
        $T_\infty$ & \unit{\kelvin} & Température ambiante hors de la machine \\
        $u$ & \unit{\cubic\meter\per\second} & Débit acoustique \\
        $v$ & \unit{\meter\per\second} & Vitesse acoustique \\
        $W_e$ & \unit{\watt} & Puissance électrique consommée par les sources acoustiques \\
        $\mathbf{e}_x$ &  & Vecteur unitaire axial du noyau \\
        $\mathbf e_{x0}$ & \echaf{---} & \echaf{définir} \\
        $Q_c$ & \unit{\watt} & Flux de chaleur extrait par l'eau circulant dans l'échangeur ambiant \\
        $Q_f$ & \unit{\watt} & Flux de chaleur apporté par les résistances chauffantes dans l'échangeur froid\\
    \end{longtable}

\bigskip

    \begin{longtable}{p{.1\textwidth} p{.1\textwidth} p{.7\textwidth}}
%    	\hline
        \multicolumn{3}{c}{Lettres grecques \textcolor{red}{ordre alphabétique ?}}  \\\hline
        \textbf{Symbole} & \textbf{Unité} & \textbf{Définition} \\\hline\hline
        $\psi_v$ & \unit{\degree} & Rotation de l'axe de symétrie du réfrigérateur par rapport à l'axe horizontal\\
        $\psi_h$ & \unit{\degree} & Rotation autour de l'axe de symétrie du réfrigérateur \\
        $\varphi_{2-1}$ & \unit{\degree} & Déphasage entre les sources acoustiques, $\varphi_{1-2} = \varphi_2 - \varphi_1$\\
        $\Phi$ & \unit{\percent} & Porosité du régénérateur \\
        $\rho_0$ & \unit{\kilo\gram\per\cubic\meter} & Masse volumique moyenne du gaz \\
        $\rho_{s}$ & \unit{\kilo\gram\per\cubic\meter} & Masse volumique du solide poreux\\
        $\delta_{\kappa,\nu}$ & \unit{\meter} & Couche limites thermique/visqueuse \\
        $\theta$ & \unit{\kelvin} & Différence entre les températures locale dans le réfrigérateur et initiale, $\theta=T_0-T_\infty$\\
        $\mu$ & \unit{\pascal\second} & Viscosité dynamique \\
        $\nu$ & \unit{\square\meter\per\second} & Viscosité cinématique \\
        $\kappa$ & \unit{\square\meter\per\second} & Diffusivité thermique \\
        $\tau_0$ & \unit{\second} & Période du signal $\tau_0 = 1/f_0$ \\
        $\xi_0$ &  & Déplacement particulaire dans le régénérateur \\
        $\xi_1$ &  & Déplacement du piston de la source acoustique principale \\
        $\xi_2$ &  & Déplacement du piston de la source acoustique secondaire \\
        $\lambda$ &  & Longueur d'onde du signal acoustique \\\hline
    \end{longtable}

\bigskip

    \begin{longtable}{p{.1\textwidth} p{.8\textwidth}}
%		\hline
        \multicolumn{2}{c}{Indices et exposants}  \\\hline
        \textbf{Symbole} & \textbf{Définition} \\\hline\hline \endfirsthead
        \multicolumn{2}{c}{Indices et exposants (suite)}  \\\hline
        \textbf{Symbole} & \textbf{Définition} \\\hline\hline \endhead
		
		\hline
		\multicolumn{2}{r}{Continué sur la page suivante...} \endfoot
        \hline \endlastfoot
		$\square_c$ & Caractéristique critique\\
		$\square^{\perp \mathbf g}$ & Perpendiculaire à la gravité $\mathbf g$ \\
		$\square^{//~\mathbf g}$ & Parallèle à la gravité $\mathbf g$ \\
		$\square^*$ & Conjugué complexe\\
		$\square_p$ & Lié au matériau poreux\\		
		$\square_g$ & Relatif au gaz dans le matériau poreux\\
		$\square_s$ & Relatif au solide du matériau poreux\\
		\hline
	\end{longtable}
\end{center}