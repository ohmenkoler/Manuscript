\chapter{Index des notations}\label{chap:IndexNotations}%\addcontentsline{toc}{chapter}{\nameref{chap:IndexNotations}}

\textcolor{red}{Taille de la 2\ieme{} colonne pour remplir la page + reparcourir manuscript pour vérifier que c'est complet et cohérent}

\begin{center}
    \begin{tabular}{ll}
%    	\hline
        \multicolumn{2}{c}{Lettres latines}  \\\hline
        \textbf{Symbole} & \textbf{Définition} \\\hline\hline
        $\mathbf{g}$ & Constante de gravité, $\mathbf{g}=\qty{9.81}{\meter\per\second\squared}$ \\
        $g$ & Gain thermoacoustique \\
        $\Grashof$ & Nombre de Grashof \\
        $k_0$ & Nombre d'onde sans perte \\
        $c$ & Célérité du son dans le milieu \\
        $k$ & Conductivité thermique \\
        $K_p$ & Perméabilité hydraulique d'un matériau poreux \\
        $C_p$ & Capacité calorifique du gaz à pression constante \\
        $C_s$ & Capacité calorifique du solide poreux \\
        $C_v$ & Capacité calorifique du gaz à volume constant \\
        $L_{\sf reg}$ & Longueur du régénérateur, $L_{\sf reg}=\qty{39}{\milli\meter}$ \\
        $\Nusselt$ & Nombre de Nusselt \\
        $P_0$ & Pression statique dans le réfrigérateur \\
        $p$ & Pression acoustique \\
        $\Peclet$ & Nombre de Péclet \\
        $\Prandtl$ & Nombre de Prandtl \\
        $\mathbf{e}_r$ & Coordonnée radiale dans le noyau \\
        $R_{\sf reg}$ & Rayon du régénérateur, $R_{\sf reg}=\qty{74}{\milli\meter}$ \\
        $\Rayleigh$ & Nombre de Rayleigh \\
        $r_h$ & Rayon hydraulique des pores du régénérateur \\
        $T_0$ & Température moyenne locale du gaz dans le noyau \\
        $T_\infty$ & Température ambiante hors \textcolor{red}{du noyau / de la machine} \\
        $u$ & Débit acoustique \\
        $v$ & Vitesse acoustique \\
        $\mathbf{e}_x$ & Coordonnée axiale dans le noyau \\
        $Q_c$ & Flux de chaleur extrait par l'eau circulant dans l'échangeur ambiant \\
        $Q_f$ & Flux de chaleur apporté par les résistances chauffantes dans l'échangeur froid\\\hline
    \end{tabular}
\end{center}

\newpage

\begin{center}
    \begin{tabular}{ll}
%    	\hline
        \multicolumn{2}{c}{Lettres grecques \textcolor{red}{ordre alphabétique ?}}  \\\hline
        \textbf{Symbole} & \textbf{Définition} \\\hline\hline
        $\psi_v$ & Rotation de l'axe de symétrie du réfrigérateur par rapport à l'axe horizontal\\
        $\psi_h$ & Rotation autour de l'axe de symétrie du réfrigérateur \\
        $\varphi_{2-1}$ & Déphasage entre les sources acoustiques, $\varphi_{1-2} = \varphi_2 - \varphi_1$\\
        $\Phi$ & Porosité du régénérateur \\
        $\rho_0$ & Masse volumique moyenne du gaz \\
        $\rho_{s}$ & Masse volumique du solide poreux\\
        $\delta_{\kappa,\nu}$ & Couche limites thermique/visqueuse \\
        $\theta$ & Différence entre les températures locale dans le réfrigérateur et initiale, $\theta=T_0-T_\infty$\\
        $\mu$ & Viscosité dynamique \\
        $\nu$ & Viscosité cinématique \\
        $\kappa$ & Diffusivité thermique \\
        $\tau_0$ & Période du signal $\tau_0 = 1/f_0$ \\
        $\xi_0$ & Déplacement particulaire dans le régénérateur \\
        $\xi_1$ & Déplacement du piston de la source acoustique principale \\
        $\xi_2$ & Déplacement du piston de la source acoustique secondaire \\\hline
    \end{tabular}
\end{center}

\bigskip

\begin{center}
	\begin{tabular}{ll}
%		\hline
		\multicolumn{2}{c}{Indices et exposants}  \\\hline
		\textbf{Symbole} & \textbf{Définition} \\\hline\hline
		$\bullet^{\perp \mathbf g}$ & Perpendiculaire à la gravité $\mathbf g$ \\
		$\bullet^{//~\mathbf g}$ & Parallèle à la gravité $\mathbf g$ \\
		$\bullet^*$ & Conjugué complexe\\
		$\bullet_p$ & Lié au matériau poreux\\		
		$\bullet_g$ & Relatif au gaz dans le matériau poreux\\
		$\bullet_s$ & Relatif au solide du matériau poreux
	\end{tabular}
\end{center}