\chapter{Récapitulatif des conditions expérimentales}
Les conditions expérimentales du chapitre~\ref{chap:EtudeExpe_ConvNat} sont résumées dans le tableau~\ref{tab:RecapCondExpe}.

\begin{table}[!ht]
\centering
	\begin{tabular}{ llll llll llll ll}
	\hline
	$\xi_1$ & $\xi_2$ & $\varphi_{2-1}$  & $DR$ & $p$ & $|Z_{ac}^{(1)}|$ & $|Z_{ac}^{(2)}|$& $\angle Z_{ac}^{(1)}$  & $\angle Z_{ac}^{(2)}$ & $T_a$  & $T_c$ & $\dot Q_a$ & $\dot Q_c$  & \multirow{2}{*}{Orientation} \\%
	
	\multicolumn{2}{c}{[\unit{\milli\meter}]} & [\unit{\degree}] & [\unit{\percent}] & [\unit{\pascal}] & \multicolumn{2}{c}{[\unit{\pascal\second\per\meter}]} & \multicolumn{2}{c}{[\unit{\degree}]}  & \multicolumn{2}{c}{[\unit{\degreeCelsius}]} &\multicolumn{2}{c}{[\unit{\watt}]} &  \\\hline\hline
	 &  & \num{-60} & \num{0} & && & & &  & \num{} & & & \multirow{4}{*}{`\texttt{H1}'} \\
	 &  & \num{-60} & \num{.7} & && & & &  & \num{} & & &  \\
	 &  & \num{-60} & \num{1.7} & & &&&  &  & \num{} & & & \\
	 &  & \num{-60} & \num{3.6} & & & &&&  & \num{} & & &  \\	 
	 &  & \num{-60} & \num{0} & && & & &  & \num{} & & & \multirow{4}{*}{`\texttt{H2}'} \\
	 &  & \num{-60} & \num{.7} & & & &&& & \num{} & & & \\
	 &  & \num{-60} & \num{1.7} & & &&& & & \num{} & & & \\
	 &  & \num{-60} & \num{3.6} & & & &&& & \num{} & & & \\
	 &  & \num{-60} & \num{0} & && & & &  & \num{} & & & \multirow{4}{*}{`\texttt{V1}'} \\
	 &  & \num{-60} & \num{.7} & & & & &&& \num{} & & & \\
	 &  & \num{-60} & \num{1.7} &  & & &&& & \num{} & & & \\
	 &  & \num{-60} & \num{3.6} &  & & &&& & \num{} & & & \\
	 &  & \num{-60} & \num{0} & && & & &  & \num{} & & & \multirow{4}{*}{`\texttt{V2}'} \\
	 &  & \num{-60} & \num{.7} & & && & && \num{} & & & \\
	 &  & \num{-60} & \num{1.7} & & & &&& & \num{} & & & \\
	 &  & \num{-60} & \num{3.6} &  & & & & \num{} & & & \\
	 \hline
	
	\end{tabular}
	\caption{Tableau récapitulatif des conditions expérimentales.\textcolor{red}{Pression ac  ?} Les déplacements $\xi$ sont exprimés en valeur ``crête'', les températures en \textbf{gras} sont mesurées au centre du noyau, celles en \textit{italique} sont le résultat d'une moyenne sur la section.\textcolor{red}{AJOUTER HEAT ONLY}}
	\label{tab:RecapCondExpe}
\end{table}

%\begin{table}[!ht]
%\centering
%	\begin{tabular}{ccccccccc}
%		\hline
%		$\xi_1$ [\unit{\mm}] & $\xi_2$ [\unit{\mm}] & $\varphi_{2-1}$ [\unit{\degree}] & $DR$ [\unit{\percent}] & $T_a$ [\unit{\degreeCelsius}] & $\dot Q_a$ [\unit{\W}] & $T_c$ [\unit{\degreeCelsius}] & $\dot Q_c$ [\unit{\W}] & Orientation \\\hline\hline
%		\num{0} & \num{0} & -- & \num{0} & & & & & V1 \\
%		\num{0} & \num{0} & -- & \num{0} & & & & & V2\\%\hline
%		\num{1} & \num{.28} & \num{-60} & \num{.4} & \textbf{\num{15.1}} | \textit{\num{15.1}} & \num{1.29} & \textbf{\num{10.9}} | \textit{\num{13}} & \num{0} & V1 \\
%		\num{1.1} & \num{.30} & \num{-60} & \num{.4} & \textbf{\num{15.3}} | \textit{\num{14.2}}& \num{.11} & \textbf{\num{12.3}} | \textit{\num{13.4}} & \num{0} & V2 \\
%		\num{1} & \num{.28} & \num{-60} & \num{.4} & \textbf{\num{16.7}} | \textit{\num{16}}& \num{-.40} & \textbf{\num{16.2}} | \textit{\num{17}} & \num{10} & V1 \\
%		\num{1.1} & \num{.32} & \num{-60} & \num{.4} &\textbf{ \num{16}} | \textit{\num{14.6}} & \num{4.47} & \textbf{\num{15.9}} | \textit{\num{16.9}} & \num{6.43} & V2 \\%\hline
%		\num{4} & \num{.89} & \num{-60} & \num{1.7} & \textbf{\num{10.6}} | \textit{\num{17.3}} & \num{33.8} & \textbf{\num{-8.82}} | \textit{\num{-1.92}} & \num{0} & V1 \\
%		\num{5} & \num{1.14} & \num{-60} & \num{1.7} & \textbf{\num{18.65}} | \textit{\num{16.5}} & \num{75} & \textbf{\num{-14.24}} | \textit{\num{-9.85}} & \num{0} & V2 \\
%		\num{4} & \num{.87} & \num{-60} & \num{1.7} & \textbf{\num{28}} | \textit{\num{27.6}} & \num{67.54} & \textbf{\num{27.83}} | \textit{\num{30.12}}& \num{135} & V1 \\
%		\num{5} & \num{1.1} & \num{-60} & \num{1.7} & \textbf{\num{31.59}} | \textit{\num{24.06}} & \num{143.94} & \textbf{\num{31.7}} | \textit{\num{36.1}} & \num{219} & V2 \\%\hline
%		\num{8.3} & \num{1.63} & \num{-60} & \num{3.5} & \textbf{\num{30.12}} | \textit{\num{30.98}} & \num{175.03} & \textbf{\num{-20.47}} | \textit{\num{-14.53}} & \num{0} & V1 \\
%		\num{8.4} & \num{1.68} & \num{-60} & \num{3.5} & \textbf{\num{22.33}} | \textit{\num{18.85}} & \num{217.72} & \textbf{\num{-24.53}} | \textit{\num{-16.81}} & \num{0} & V2 \\
%		\num{8.3} & \num{1.56} & \num{-60} & \num{3.5} & \textbf{\num{32.58}} | \textit{\num{35.02}} & \num{240.56} &\textbf{\num{-6.03}} | \textit{\num{2.8}}  & \num{112} & V1 \\
%		\num{8.3} & \num{1.66} & \num{-60} & \num{3.5} & \textbf{\num{27.28}} | \textit{\num{22.24}} & \num{254.95} & \textbf{\num{-6.6}} | \textit{\num{-1.22}} & \num{112} & V2 \\
%		\num{8.4} & \num{1.56} & \num{-60} & \num{3.5} & \textbf{\num{33.51}} | \textit{\num{36.85}} & \num{264.78} & \textbf{\num{2.08}} | \textit{\num{11.77}} & \num{219} & V1 \\
%		\num{8.3} & \num{1.63} & \num{-60} & \num{3.5} & \textbf{\num{30.71}} | \textit{\num{24.89}} & \num{290.28} & \textbf{\num{3.87}} | \textit{\num{11.64}} & \num{219} & V2 \\
%		\num{8.3} & \num{1.54} & \num{-60} & \num{3.5} & \textbf{\num{38.5}} | \textit{\num{40.92}} & \num{303.34} & \textbf{\num{16.36}} | \textit{\num{25.87}} & \num{330} & V1 \\
%		\num{8.3} & \num{1.62} & \num{-60} & \num{3.5} & \textbf{\num{36.35}} | \textit{\num{28.01}} & \num{304.32} & \textbf{\num{16.77}} | \textit{\num{24.67}} & \num{330} & V2 \\
%		\num{8.3} & \num{1.54} & \num{-60} & \num{3.5} & \textbf{\num{40.59}} | \textit{\num{42.37}} & \num{298.5} & \textbf{\num{28.57}} | \textit{\num{37.68}} & \num{446} & V1 \\
%		\num{8.4} & \num{1.6} & \num{-60} & \num{3.5} & \textbf{\num{35.55}} | \textit{\num{29.27}} & \num{333.64} & \textbf{\num{29.52}} | \textit{\num{39.74}} & \num{446} & V2 \\%\hline
%%%		\num{0} & \num{0} & - & \num{0} & & & & & H1 \\
%%%		\num{0} & \num{0} & - & \num{0} & & & & & H2 \\%\hline
%%		\num{8.5} & \num{} & \num{-60} & \num{3.5} & \textbf{\num{}} | \textit{\num{}} & \num{} & \textbf{\num{}} | \textit{\num{}} & \num{0} & H1 \\
%%		\num{8.5} &\num{}  & \num{-60} & \num{3.5} & \textbf{\num{}} | \textit{\num{}} & \num{} & \textbf{\num{}} | \textit{\num{}} & \num{0} & H2 \\
%%		\num{8.5} & \num{} & \num{-60} & \num{3.5} & \textbf{\num{}} | \textit{\num{}} & \num{} & \textbf{\num{}} | \textit{\num{}} & \num{1212} & H1 \\
%%		\num{8.5} & \num{} & \num{-60} & \num{3.5} & \textbf{\num{}} | \textit{\num{}} & \num{} & \textbf{\num{}} | \textit{\num{}} & \num{1212} & H2 \\
%\hline
%	\end{tabular} %\textbf{\num{}} | \textit{\num{}}
%	\caption{Tableau récapitulatif des conditions expérimentales.\textcolor{red}{Pression ac  ?} Les déplacements $\xi$ sont exprimés en valeur ``crête'', les températures en \textbf{gras} sont mesurées au centre du noyau, celles en \textit{italique} sont le résultat d'une moyenne sur la section.}
%	\label{tab:RecapCondExpe}
%\end{table}


