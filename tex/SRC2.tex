\chapter{Optimisation de la source acoustique secondaire}

\section{Déphasage entre les sources}

Le déphasage entre les sources acoustique est primordial pour assurer l'efficacité du réfrigérateur. Pour cela, des accéléromètres sont placés sur leur équipage mobile pour l'estimer au mieux.  

\textcolor{red}{Ajouter définition des conventions et axes des $x$}

\begin{figure}[!ht]
    \centering
    \external{fig_AcceleroDefinition}
    %\externalremake
    \begin{tikzpicture}
    \def\LX{1};
    \def\LY{2};
    \def\CoreX{1.5};
    \def\CoreY{.9*\LY};
    
    \draw[line width=.5mm] (-2.5*\LX,0) to[out=90,in=-180] (-\LX,\LY) -- ++(2*\LX,0) -- ++(.5*\LX,-2*\LY/3) -- ++(.2*\LX,0) -- ++(0,2*\LY/3);
\draw[line width=.5mm] (\LX,\LY) -- ++(\LX,0) to[out=0,in=90] (3.5*\LX,0);

\fill[left color=red,right color=blue] ({-\LX+.4*\CoreX},0) rectangle ({-\LX+.9*\CoreX},\CoreY);
\draw[line width=.5mm] (-\LX,\CoreY) -- ++(\CoreX,0);
\fill[PythonOrange] (-.7*\LX,0) rectangle ++(.1,.05) node(a2)[midway]{};
\draw[fill=PythonBlue] (-.9*\LX,0) -- ++(0,\CoreY) to[out=-80,in=90] (-.7*\LX,0);
\draw ({-\LX+.4*\CoreX},0) -- ++(0,\CoreY);
\draw ({-\LX+.9*\CoreX},0) -- ++(0,\CoreY);

\fill[PythonOrange] (2.95*\LX,0) rectangle ++(.1,.05) node(a1)[midway]{};
\draw[fill=PythonBlue] (1.6*\LX,0) |- ++(.3*\LX,.9*\LY/3) |- ++(\LX,.2*\LY) arc (90:0:.05) -- ++(0,-.5*\LY);
    
    \begin{scope}[xscale=1,yscale=-1]
        \draw[line width=.5mm] (-2.5*\LX,0) to[out=90,in=-180] (-\LX,\LY) -- ++(2*\LX,0) -- ++(.5*\LX,-2*\LY/3) -- ++(.2*\LX,0) -- ++(0,2*\LY/3);
\draw[line width=.5mm] (\LX,\LY) -- ++(\LX,0) to[out=0,in=90] (3.5*\LX,0);

\fill[left color=red,right color=blue] ({-\LX+.4*\CoreX},0) rectangle ({-\LX+.9*\CoreX},\CoreY);
\draw[line width=.5mm] (-\LX,\CoreY) -- ++(\CoreX,0);
\fill[PythonOrange] (-.7*\LX,0) rectangle ++(.1,.05) node(a2)[midway]{};
\draw[fill=PythonBlue] (-.9*\LX,0) -- ++(0,\CoreY) to[out=-80,in=90] (-.7*\LX,0);
\draw ({-\LX+.4*\CoreX},0) -- ++(0,\CoreY);
\draw ({-\LX+.9*\CoreX},0) -- ++(0,\CoreY);

\fill[PythonOrange] (2.95*\LX,0) rectangle ++(.1,.05) node(a1)[midway]{};
\draw[fill=PythonBlue] (1.6*\LX,0) |- ++(.3*\LX,.9*\LY/3) |- ++(\LX,.2*\LY) arc (90:0:.05) -- ++(0,-.5*\LY);
    \end{scope}
    
    \draw[->,green] ({-.7*\LX},.1*\LY) -- ++(.5,0) node[pos=1,above right]{$\xi_2$};
    \draw[->,green] (1.6*\LX,0) -- ++(-.5,0)node[pos=1,below]{$\xi_1$};
    
    \foreach \a in {1,2}{
    \draw[->,PythonOrange,thick] (6*\LX,-1) node[above]{Accéléromètres} to[out=-120,in=-45] (a\a.south east);
    \draw[->,PythonBlue,thick] (6*\LX,.5) node[below]{sources acoustiques} to[out=120,in=45] ($(a\a.north east)+(0,.5)$);
    }
    
    
    %\draw[line width=1mm] (5*\LX,.5*\LY) -- ++(6*\LX,0);
    %\draw[line width=1mm] (5*\LX,-.5*\LY) -- ++(6*\LX,0);
    %
    %\draw (-2.5*\LX,\LY) node[above]{\bf (a)};
    %\draw (5*\LX,\LY) node[above]{\bf (b)};
    
\end{tikzpicture}
    \caption{Emplacement des accéléromètres (en \textcolor{PythonOrange}{orange}) sur les sources acoustiques (en \textcolor{PythonBlue}{bleu}). Les flèches en \textcolor{PythonRed}{rouge} dénotent le sens de l'excursion positive des sources acoustiques. \textcolor{red}{C'est clair ?}}
    \label{../fig:AcceleroDefinition}
\end{figure}

\begin{figure}[ht!]
    \centering
    \external{fig_PhaseGBFOscPiston}
%    \externalremake
    \begin{tikzpicture}
    \def\width{.7\textwidth};
    \def\height{\width};
    \def\spx{3cm};
    \def\lab{.5cm};
    
    % \begin{axis}[name=GBFOsc,height=\height,width=\width,
    % domain=-180:180, 
    % xtick={-180,0,180},xticklabels={$-\pi$,$0$,$\pi$},
    % ytick={-360,-180,0,180},yticklabels={$-2\pi$,$-\pi$,$0$,$\pi$},
    % xlabel={$\phi_{\sf gbf}$},ylabel={$\phi_{osc}$},
    % legend style={at={(1,1.05)},anchor=south east},
    % ]
    % \addplot[only marks,draw=PythonBlue] file {../fig/fig_PhaseGBFOscPiston/data/data_PhaseGBFOsc.txt};
    % \addplot[dashed,draw=PythonBlue] {-x-180};
    % \legend{Exp.\\ $\phi_{\sf osc}=-\phi_{\sf gbf}-180$\\};
    % \end{axis}
    
    % \begin{axis}[name=OscPiston,height=\height,width=\width,
    % domain=-180:180, xtick={-180,0,180},xticklabels={$-\pi$,$0$,$\pi$},
    % ytick={-180,0,180},yticklabels={$-\pi$,$0$,$\pi$},
    % xlabel={$\phi_{\sf osc}$},ylabel={$\phi_{pist}$},
    % at={($(GBFOsc.east)+(\spx,0)$)},anchor=west,
    % ]
    % \addplot[only marks,draw=PythonBlue] file{../fig/fig_PhaseGBFOscPiston/data/data_PhaseOscPiston.txt};
    % \addplot[dashed,draw=PythonBlue] {-x};
    % \end{axis}
    
    % \draw ($(GBFOsc.north west)+(\lab,\lab)$) node[]{\bf (a)};
    % \draw ($(OscPiston.north west)+(\lab,\lab)$) node[]{\bf (b)};
    
    \begin{axis}[name=GBFOsc,height=\height,width=\width,
    domain=-180:180, 
    xtick={-180,0,180},xticklabels={$-\pi$,$0$,$\pi$},
    ytick={-360,-180,0,180},yticklabels={$-2\pi$,$-\pi$,$0$,$\pi$},
    xlabel={$\phi^{\sf gbf}_{\sf RIX-HP}$ [\unit{\radian}]},ylabel={$\phi_{\sf RIX-HP}^{\sf pist}$ [\unit{\radian}]},
    legend style={at={(.05,.95)},anchor=north west},
    ]
    \addplot[only marks,draw=PythonBlue] file {../fig/fig_PhaseGBFOscPiston/data/data_PhaseGBFPiston.txt};
    \addplot[dashed,draw=PythonBlue] {.6418*x+104.49};
    \legend{Exp.\\ $\phi_{\sf RIX-HP}^{\sf pist}=0.6\phi_{\sf RIX-HP}^{\sf gbf}+104.5$\\};
    \end{axis}
    
\end{tikzpicture}
    \caption{Déphasage entre les pistons des sources en fonction du déphasage réglé sur le générateur de signal.}
    \label{fig:PhaseGBFOscPiston}
\end{figure}

\textcolor{red}{Présenter les déphasages réglés sur le GBF, puis des formes d'ondes de tension ET de déplacement (déphasage pas toujours le même)}

\begin{figure}
    \centering
    \external{fig_GradT_PhaseHP}
%    \externalremake
    \begin{tikzpicture}
    \def\width{.7\textwidth};
    \def\height{7cm};
    \def\xcursor{306};
    
    \begin{axis}[name=tT_WHP,height=\height,width=\width,no marks,
    xmin=0,
    xlabel={$t$ [\unit{\second}]},
    ylabel={$\dfrac{\Delta T}{\max(\Delta T)}$},
    legend style={at={(.05,.05)},anchor=south west},
    line width=1pt,
    grid=both,
    grid style={line width=.1pt, draw=gray!10},
    major grid style={line width=.2pt,draw=gray!50},
    ]
    % \foreach \i [evaluate=\i as \c using \i*5] in {5,11}{
    % %\def\colorlist{MatlabBlue,MatlabOrange,MatlabYellow,MatlabPurple,MatlabGreen,MatlabCyan,MatlabBrown};
    % \addplot file {../fig/fig_GradT_PhaseHP/data/data_tT\i.txt};
    % %\draw (0.1,\i/10) node[]{\i};
    % }
    
    \addplot[MatlabBlue] file {../fig/fig_GradT_PhaseHP/data/data_tDT_TA_norm.txt};
    \addplot[MatlabOrange] file {../fig/fig_GradT_PhaseHP/data/data_tDT_water_norm.txt};
    \draw[dashed,black!50] ({axis cs:\xcursor,0}|-{rel axis cs:0,0}) -- ({axis cs:\xcursor,0}|-{rel axis cs:0,1}) node[pos=.5,right,color=black]{\shortstack{$t=\qty{\xcursor}{\second}$\\$\phi^{\sf GBF}_{\sf RIX-HP}=\ang{-10}$}};
    \legend{Thermoacoustique \\ Eau dans l'AHX \\};
    \end{axis}    
\end{tikzpicture}
    \caption{Différences normalisées de température entre les thermocouples 5 et 11, et entre l'entrée et la sortie de l'échangeur ambiant \textcolor{red}{préciser}. Le déphasage est modifié toutes les \qty{120}{\second} selon l'ordre suivant : \qtylist{-60;-30;-10;0;10;30;60;-100;100;-180;180;-120;120}{\degree}. \textcolor{red}{Ajouter curseur de changement de phases ?}}
    \label{fig:GradT_PhaseHP}
\end{figure}

\textcolor{green}{REFAIRE LES FICHIERS DE MESURES LES NUMEROS SONT DECALES DE 1}

\section{Influence de la puissance d'alimentation de la source secondaire}



\textcolor{red}{Profils entre avec et sans source secondaire (RIX à faible amplitude, à refaire à haute amplitude ?), $\Delta T_{TA}$ diminue beaucoup quand le HP est éteint : impact pas si nul que ça. Aussi, évolution temporelle quand on allume la source à 7 V et un déphasage de 0 rad GBF (Combien au niveau des pistons ?)}


\begin{figure}
    \centering
    \external{fig_GradT_PuissHP}
    %\externalremake
    \begin{tikzpicture}
    \def\width{.7\textwidth};
    \def\height{7cm};
    
    \begin{axis}[name=tT_WHP,height=\height,width=\width,no marks,
    %xtick={-180,0,180},xticklabels={$-\pi$,$0$,$\pi$},
    %ytick={-360,-180,0,180},yticklabels={$-2\pi$,$-\pi$,$0$,$\pi$},
    xlabel={$t$ [\unit{\second}]},
    ylabel={$T$ [\unit{\degreeCelsius}]},
    legend style={at={(1,1.05)},anchor=south east},
    line width=1pt,
    grid=both,
    grid style={line width=.1pt, draw=gray!10},
    major grid style={line width=.2pt,draw=gray!50},
    ]
    \foreach \i [evaluate=\i as \c using \i*5] in {5,11}{
    %\def\colorlist{MatlabBlue,MatlabOrange,MatlabYellow,MatlabPurple,MatlabGreen,MatlabCyan,MatlabBrown};
    \addplot file {../fig/fig_GradT_PuissHP/data/data_tT\i.txt};
    %\draw (0.1,\i/10) node[]{\i};
    }
    \end{axis}    
\end{tikzpicture}
    \caption{Température aux points chaud et froid avant et après allumage de la source secondaire \textcolor{red}{Préciser les paramètres d'alim du HP | Tracer les autres points}}
    \label{fig:GradT_PuissHP}
\end{figure}